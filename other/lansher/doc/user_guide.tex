\documentclass[11pt]{article}
\usepackage[utf8]{inputenc}
\usepackage[T1]{fontenc}
\usepackage{indentfirst}

\usepackage{listings}
\usepackage[shortcuts]{extdash}
\usepackage[a4paper, total={6in, 9in}]{geometry}

\lstset{literate=%
    {á}{{\' a}}1
    {é}{{\' e}}1
    {è}{{\` e}}1
    {í}{{\' i}}1
    {ý}{{\' y}}1
    {ó}{{\' o}}1
    {ú}{{\' u}}1
    {ů}{{\r{u}}}1
    {š}{{\v s}}1
    {č}{{\v c}}1
    {ř}{{\v r}}1
    {ě}{{\v e}}1
    {ž}{{\v z}}1
    {ú}{{\' u}}1
    {ä}{{\" a}}1
    {ö}{{\" o}}1
    {ł}{{\l}}1
    {ę}{{\k e}}1
    {ą}{{\k a}}1
    {ź}{{\" o}}1
    {ś}{{\" s}}1
}
\lstset{basicstyle=\ttfamily\footnotesize,breaklines=true}


\title{\textbf{Uživatelská dokumentace LANSHER }}
\author{Vilém Zouhar}
\date{2017}

\usepackage{amsmath}
\begin{document}

\maketitle

\section{Úvod} 
LANSHER (\textbf{lan}guage distingui\textbf{sher}) slouží k rozlišení jazyka na základě velkého množství trénovacích dat. Používá se ve dvou krocích:

\begin{enumerate}
\item Vytvoření databáze z JSON formátovaného souboru
\item Použití databáze k rozeznání jazyka vstupu
\end{enumerate}

Přičemž není nutné pro každé použití absolvovat oba dva kroky. Zejména krok 1., výpočetně náročný, je možné provést jen jednou a používat tak vytvořenou databázi na rozeznávání jazyka. Je také možné použít již zpracovanou databázi dodávanou s programem.

\section{Rychlé použití}
Je možné použít předpřipravenou databázi (10 jazyků z různých překladů \textit{Alenky v říši divů}) například následujícím příkazem:

\begin{lstlisting}
./lansher.py -ld alice_full.ld -i "Byl pozdní večer - první máj.."
\end{lstlisting}

Srovnání jazyků dosáhneme za pomocí:


\begin{lstlisting}
./lansher.py -ld alice_full.ld -lc
\end{lstlisting}

\section{Vytváření databáze}
Databáze se vytváří z JSON souboru v daném formátu: (soubor \textit{alenka.json})
\begin{lstlisting}[postbreak=, breakautoindent=true, breakindent=0pt, breaklines]
{
"cs":
	"Alenka ani chvíli nemeškala a vskočila za ním, aniž jen zdaleka pomyslila, jak se kdy opět dostane ven.",
"en": 
	"In another moment down went Alice after it, never once considering how in the world she was to get out again.",
"de":
	"Den nächsten Augenblick war sie ihm nach in das Loch hineingesprungen, ohne zu bedenken, wie in aller Welt sie wieder herauskommen könnte.",
"pl":
	"Wczołgała się więc za nim do króliczej nory nie myśląc o tym, jak się później stamtąd wydostanie."
}
\end{lstlisting}

Není třeba definovat všechny jazyky, a ani není nutné dodržovat dvouznakové značení. Pro přehlednost se v tomto dokumentu budeme dvou znaků držet. JSON vstup se do programu vkládá pomocí přepínače \textbf{\=/ls}, nebo \textbf{\=/\=/language\=/samples}.

\section{Ukládání databáze}
Pokud do programu vkládáme text ke zpracování, což je obecně náročná výpočetní operace, můžeme si takto zpracovanou databázi uložit za pomocí přepínače \textbf{\=/sd}, nebo \textbf{\=/\=/save\=/database}.

Pokud třeba načítáme soubor \textit{alenka.json} a chceme zpracovanou verzi uložit do souboru \textit{alenka.ld}, použijeme přepínače \textbf{\=/ls} a \textbf{\=/sd}:
\begin{lstlisting}
./lansher.py -ls alenka.json -sd alenka.ld
\end{lstlisting}

\section{Načítání databáze}
Je-li k dispozici již vytvořená databáze, do programu ji načteme pomocí \textbf{\=/ld}, nebo \textbf{\=/\=/load\=/database}. V příkladu:
\begin{lstlisting}
./lansher.py -ld alenka.ld
\end{lstlisting}

Tento příkaz však sám o sobě nemá žádný výstup, neboť mu nebyla předán žádný přepínač pro výstupní operace.

\section{Slučování databáze}
Jelikož existují dvě možnosti jak vytvářet databázi, je možné je zkombinovat a vytvořit tak novou, která je sloučením obou dvou. Toho docílíme načtením již vytvořené databáze, zpracováním nového úseku textu (\textit{kralik.json}) a uložením do souboru (\textit{alenka\_kralik.ld}):

\begin{lstlisting}
./lansher.py -ls kralik.json -ld alenka.ld -sd alenka_kralik.ld
\end{lstlisting}


\section{Vstup vzorku}
Má-li program načtenou databázi (např. z předem zpracovaného textu), můžeme mu vložit vzorek k rozeznání.

Pokud chceme předat vzorek z příkazové řádky, použijeme přepínač \textbf{\=/i}, nebo \textbf{\=/\=/input}:
\begin{lstlisting}
./lansher.py --load-database alenka.ld --input "Alenka v říši divů"
\end{lstlisting}

Výstup:
\begin{lstlisting}
input text comparison:
cs (96.988%)
en (1.045%)
pl (1.023%)
de (0.944%)
\end{lstlisting}


Pokud chceme předat vzorek ze souboru, použijeme přepínač \textbf{\=/fi}, nebo \textbf{\=/\=/file\=/input}. \textit{vzorek\_textu.txt} obsahuje řetězec: \textit{Sie könnte nicht mehr herauskommen.}
\begin{lstlisting}
./lansher.py --load-database alenka.ld --file-input vzorek_textu.txt
\end{lstlisting}

Výstup:
\begin{lstlisting}
input text comparison:
de (98.662%)
en (0.474%)
pl (0.443%)
cs (0.421%)
\end{lstlisting}

Databáze je v našem případě velmi malá. Pro přesnější rozlišování je třeba mít rozsah v rámci několika knih či slovníků.

\section{Srovnání jazyků}
Má-li program načtenou databázi, můžeme porovnat jazyky mezi sebou za pomocí přepínače \textbf{\=/lc}, nebo \textbf{\=/\=/language\=/comparison}, což pro každý jazyk vypíše nejbližší jiný:
\begin{lstlisting}
./lansher.py --load-database alenka.ld --language-comparison
\end{lstlisting}

Výstup:
\begin{lstlisting}
cs: pl (97.92%)
de: en (97.33%)
pl: cs (97.66%)
en: de (96.97%)
\end{lstlisting}

\section{Přehled příkazů}
\begin{center}
\begin{tabular}{ l l l }
 \textbf{\=/i}, \textbf{\=/\=/input} 				& & vzorek z příkazové řádky \\ 
 \textbf{\=/fi}, \textbf{\=/\=/file\=/input} 		& & vzorek ze souboru \\  
 \textbf{\=/ls}, \textbf{\=/\=/language\=/samples} 	& & vytvoření databáze \\
 \textbf{\=/ld}, \textbf{\=/\=/load\=/database} 	& & načtení připravené databáze \\
 \textbf{\=/sd}, \textbf{\=/\=/save\=/database} 	& & uložení načtené databáze \\
 \textbf{\=/v}, \textbf{\=/\=/version} 				& & aktuální verze programu \\
 \textbf{\=/h}, \textbf{\=/\=/help} 				& & nápověda \\
 & \hspace{100pt} &  \\
\end{tabular}
\end{center}

\end{document}
