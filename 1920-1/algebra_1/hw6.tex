% !TeX spellcheck = cs_CZ

\documentclass[a4paper]{article}
\usepackage[english]{babel}
\usepackage[utf8x]{inputenc}
\usepackage[T1]{fontenc}
\usepackage{listings}
\usepackage[a4paper,margin=2cm]{geometry}
\usepackage{amsmath}
\usepackage{graphicx}
\usepackage[colorlinks=true, allcolors=blue]{hyperref}
\usepackage{wasysym} % smileys
\usepackage{fancyhdr}
\setlength\parindent{0pt} % indent

% my commands:
\newcommand{\n}{\newline}
\newcommand{\tab}{\hspace{1cm}}

\begin{document}

\thispagestyle{fancy} % beware the difference between \thispagestyle and \pagestyle
\lhead{6th homework, 19-11-2019}
\rhead{Vilém Zouhar}

% \textbf{Lemma 1} Cyclic groups are abelian.

% \textbf{Proof} 

% \begin{align*}
%	& \forall x, y \in C: x = a^i, y = a^j, x\cdot y = a^i \cdot a^j = a^{i+j} = a^j \cdot a^i = y \cdot x
%\end{align*}

%\vspace{1cm}

%\textbf{Statement} $S_4/V \cong S_3$

%\textbf{Proof}

\begin{align*}
	& V = \{id, (12)(34), (13)(24), (14)(23)\} \\
	& |S_4| = 24, |V| = 4 \\
	& |S_4/V| = 6 \\
\end{align*}

There are two groups of the order of $6$. Namely $C_6$ and $S_3$. By showing that $S_4/V$ is not isomorphic to $C_6$, we actually show that it is isomorphic to $S_3$.

For $C_6$ it is true, that elements are of order 6: $\forall x \in C_6: x^6 = x^1$ and $x \ne x^i \forall i \in \{2,3,4,5\}$. This is not true for $S_4$, because it contains elements of maximum order of $4$. Since $V \subseteq S_4$, then there cannot be an element of order 6 in $S_4/V$.

\end{document}