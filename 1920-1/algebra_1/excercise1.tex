% !TeX spellcheck = cs_CZ

\documentclass[a4paper]{article}
\usepackage[english]{babel}
\usepackage[utf8x]{inputenc}
\usepackage[T1]{fontenc}
\usepackage{listings}
\usepackage[a4paper,margin=2cm]{geometry}
\usepackage{amsmath}
\usepackage{amssymb}
\usepackage{graphicx}
\usepackage[colorlinks=true, allcolors=blue]{hyperref}
\usepackage{wasysym} % smileys
\usepackage{fancyhdr}
\setlength\parindent{0pt} % indent

% my commands:
\newcommand{\n}{\newline}
\newcommand{\tab}{\hspace{1cm}}

\begin{document}

\thispagestyle{fancy} % beware the difference between \thispagestyle and \pagestyle
\lhead{1st excercise, 30-10-2019}
\rhead{Vilém Zouhar}

\section{}
\begin{align*}
	\forall a, &b \in G: \\
	a^2b^2 &= (ab)^2 \\
	aabb &= abab \\
	a^{-1}aabbb^{-1} &= a^{-1}abab^{-1} \\
	ab &= ba \\
\end{align*}

Thus we can conclude, that $G$ is abelian.

\section{}

\subsection*{Intersection}

\subsubsection*{Identity}

Since $e \in H$ and $e \in K$, then $e \in H \cap K$.

\subsubsection*{Inverse}

$\forall x \in H \cap K: x \in H, x \in K \Rightarrow x^{-1} \in H, x^{-1} \in K \Rightarrow x^{-1} \in H \cap K$.

\subsubsection*{Closure under multiplication}

$\forall x,y \in H \cap K: x,y \in H, x,y \in K \Rightarrow xy \in H, xy \in K \Rightarrow xy \in H \cap K$.


Thus we can conclude, that $H\cap K$ form a subgroup.

\subsection*{Union}

Consider $A = \{id, (12)\}, B = \{id, (13)\} \subseteq S_3$. Then both $A$ and $B$ are subgroups, but $(12)(13) = (123) \notin A \cup B$.

So it is not true, that $A \cup B$ must be a group.

\section{}

$S_3 = \{id, (12), (13), (23), (123), (132) \}$

\begin{tabular}{ | l | l |}
	\hline
	elements & normal \\
	\hline
	$\{id\}$ & yes \\
	\hline
	$\{id, (12)\}$ & no \\
	\hline
	$\{id, (13)\}$ & no \\
	\hline
	$\{id, (23)\}$ & no \\
	\hline
	$\{id, (123), (132)\}$ & yes \\
	\hline
	$\{id, (12), (13), (23), (123), (132)\}$ & yes \\
	\hline
\end{tabular}

\section{}

\subsection*{Subgroup}

\subsubsection*{Identity}

$\forall g \in G: ge = eg \Rightarrow e \in Z(G)$.

\subsubsection*{Closure under multiplication}
$forall a, b \in Z(G): \forall g \in G: ag = ga, ab = ba \Rightarrow g(ab) = abg = (ab)g \Rightarrow ab \in Z(G)$.

\subsubsection*{Inverse}

$\forall x \in Z(G): \forall g \in G: x^{-1}(gx)x^{-1} = x^{-1}(xg)x^{-1} \Rightarrow x^{-1}g = gx^{-1} \Rightarrow x^{-1} \in Z(G)$.


\subsection*{Normality}

$\forall g\in G: g\cdot Z(G) = Z(G) \cdot g$, since $Z(G)$ commutes with every element of $G$, but this is also the definition of normality, because every left coset is also (exactly) a right coset.

\section{}

\subsection*{Subgroup}

\newcommand{\sln}{\text{SL}_n(\mathbb{R})}
\newcommand{\gln}{\text{GL}_n(\mathbb{R})}

\subsubsection*{Identity}

$det(I_n) = 1 \Rightarrow I_n \in \sln, \forall x \in \sln: xI_n = I_nx = x$.

\subsubsection*{Closure under multiplication}
$forall a, b \in \sln: det(ab) = 1 \Rightarrow = ab \in \sln$.

\subsubsection*{Inverse}

$\forall x \in \sln: det(x^{-1}) = \frac{1}{det(x)} = 1 \Rightarrow x^{-1} \in \sln$.

\subsection*{Normality}

Using the alternative definition of normality:

$\forall g \in \gln, \forall x \in \sln: det(gxg^{-1}) = \frac{det(g)\cdot 1}{det(g)} = 1 \Rightarrow gxg^{-1} \in \sln$.

\subsection*{Quotient group}

$\gln/\sln = \{\{ x \in \gln: det(x) = r\}, r \in \mathbb{R} \backslash \{0\}\}$ (sets of matricies of given determinant)	
\end{document}