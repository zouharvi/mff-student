% !TeX spellcheck = cs_CZ

\documentclass[a4paper]{article}
\usepackage[english]{babel}
\usepackage[utf8x]{inputenc}
\usepackage[T1]{fontenc}
\usepackage{listings}
\usepackage[a4paper,margin=2cm]{geometry}
\usepackage{amsmath}
\usepackage{amssymb}
\usepackage{graphicx}
\usepackage[colorinlistoftodos]{todonotes}
\usepackage[colorlinks=true, allcolors=blue]{hyperref}
\usepackage{wasysym} % smileys
\usepackage{fancyhdr}
\setlength\parindent{0pt} % indent

% my commands:
\newcommand{\n}{\newline}
\newcommand{\tab}{\hspace{1cm}}

\begin{document}

\thispagestyle{fancy} % beware the difference between \thispagestyle and \pagestyle
\lhead{1st homework, 08 - 09 - 2019}
\rhead{Vilém Zouhar}

LD := left divisible $\forall a, b \in G: \exists x: a\cdot x = b$\\
RD := right divisible $\forall a, b \in G: \exists x: x\cdot a = b$\\
LC := left cancellative\\
RC := right cancellative\\

\textbf{Proposition:} A semigroup G is a group iff it has a unit and is LC and RD. \\

\textbf{Proof:}

$\Rightarrow$ ($G$ is a group $\rightarrow$ G has a unit and is LC and RD):\\
Since $G$ is a group, then it is also divisible and cancellative, hence is LC and RD. In this direction we only need to prove the existence of a unit.
\begin{align*}
 & \forall g,h \in G: \exists l_g, r_g, l_h, r_h: l_g \cdot g = g \cdot r_g,\ l_h \cdot g = g \cdot r_h \text{ (RD, LD)}\\
 & (g\cdot r_g) \cdot h = g\cdot h = g \cdot (l_h \cdot h) = (g \cdot l_h) \cdot h \ \text(associativity)\\
 & (g\cdot r_g) \cdot h = (g\cdot l_h) \cdot h \Rightarrow g\cdot r_g = g\cdot l_h\ \text(RC) \\
 & g\cdot r_g = g\cdot l_h \Rightarrow r_g = l_h\ \text(LC) \\
 & \text{for } g = h \text{ we denote } l_g = r_g = u_g \\
 & \text{but since } \forall g,h \in G: u_g \cdot g = g = g \cdot u_g,\ u_h \cdot g = g = g \cdot u_h \text{ and } u_g = u_h \\
 & \text{then } u_g = u_h = u \text{ is a unit of } G
\end{align*}

$\Leftarrow$ ($S$ is a semigroup with a unit and is LC and RD $\rightarrow$ it is a group):\\
We only need to prove $S$ is LD and RC for it to be a group.\\

LD:
\begin{align*}
 & \forall x \in S\ \exists x_l \in S: x_l\cdot x = u \text{ (unit of G, RD)} \\
 & \text{let } q = x\cdot x_l \\
 & q\cdot q = (x\cdot x_l)\cdot(x\cdot x_l) = x\cdot (x_l\cdot x)\cdot x_l = x\cdot (u \cdot x_l) = x\cdot x_l = q \text{ (associativity)} \\
 & \text{similarly for q: } \exists q_l\in S: q_l\cdot q = u \\
 & x \cdot x_l = q = u\cdot q = (q_l \cdot q) \cdot q = q_l \cdot (q\cdot q) = q_l \cdot q = u
\end{align*}

We've shown, that for $x \in S$ if we start from the left inverse ($x_l\cdot x = u$), then it is also the right inverse ($x\cdot x_l = u$). $\forall x, b\in S\ \exists c:\ x\cdot c = b$, let $b=x_l\cdot b$, then $x\cdot (x_l \cdot b) = u\cdot b = b$. This means, that $S$ is DL.\\

RC:
\begin{align*}
 & \forall g, a, b \in S, \text{ for which } a\cdot g = b\cdot g: \\
 & \exists g': gg' = u \text{ (LD)} \\
 & (a\cdot g)\cdot g' = (b\cdot g)\cdot g' \\
 & a\cdot(g\cdot g') = b\cdot (g\cdot g') \text { (associativity)} \\
 & a\cdot u = b \cdot u \\
 & a = b 
\end{align*}
This showed, that $\forall g, a, b \in S: (a\cdot g = b\cdot g) \Rightarrow a = b$, which is the definition of RC.

Since $S$ is also LD and RC, we can conclude, that it is a group. $\square$ \\
    
\textbf{Note:}

The first implication would hold true even if we replaced RD with LD, but this wouldn't be true for the second one. We show this with a counterexample: 
\begin{align*}
 & S = \{0, 1, 2\} \\
 & \forall a, b \in S \backslash \{0\}: a\times b = b, 0\times a = a \times 0 = a\\
 & \hspace{1cm} \text{for } a, b \in \{1,2\}:\ a\times (b\times c) = a\times c = c = b\times c = (a\times b)\times c\\
 & \hspace{1cm} \text{for } a, b, \text{ or } c = 0 \text{ associativity also holds (intuitively or by an exhaustive proof)} \\
 & 0 \text{ is obviously the unit by definition} \\
 & \forall a, b \in S: \exists x \in S: a\times x = b \text{ for } x = b \text{ (LD holds)} \\
 & \forall x, a, b \in S: x\times a = x\times b \Rightarrow a = b \text { because } x\times a = a \text{ and } x\times b = b \text{ (LC holds)} \\
 & x \times 1 = 2, \text{ has no solutions, because } x \times 1 = 1 \text{ (is not RD)}
\end{align*}
We constructed a semigroup with a unit and which is LD and LC, but which is not RD. $\square$

\end{document}
























