% !TeX spellcheck = cs_CZ

\documentclass[a4paper]{article}
\usepackage[english]{babel}
\usepackage[utf8x]{inputenc}
\usepackage[T1]{fontenc}
\usepackage{listings}
\usepackage[a4paper,margin=2cm]{geometry}
\usepackage{amsmath}
\usepackage{graphicx}
\usepackage[colorlinks=true, allcolors=blue]{hyperref}
\usepackage{wasysym} % smileys
\usepackage{fancyhdr}
\setlength\parindent{0pt} % indent

% my commands:
\newcommand{\n}{\newline}
\newcommand{\tab}{\hspace{1cm}}

\begin{document}

\thispagestyle{fancy} % beware the difference between \thispagestyle and \pagestyle
\lhead{8th homework, 03-12-2019}
\rhead{Vilém Zouhar}

\section{}

\begin{align*}
	\Phi(p^m)
	& = \#\text{numbers relatively prime to } p^m \\
	& = \#\text{numbers less than } p^m - \#\text{numbers not relatively prime to } p^m \\
	& = |\{0, 1, 2, \ldots, p^m-1\}| - \#\text{numbers with $p$ in factor decomposition} \\
	& = p^m - |\{0, p, 2p, 3p, \ldots, \ldots\}| \\
	& = p^m - p^m/p = p^m - p^{m-1}
\end{align*}

\section{}

We can prove that $\Phi(ab) = \Phi(a)\Phi(b)$ by showing, that there exists a mapping between $Z_{ab}$ and $Z_a \times Z_b$. Such function $\alpha: Z_{ab} \rightarrow Z_a \times Z_b $ can be: $\alpha(x) = (x \mod a, x \mod b)$.

If $\alpha(x) = \alpha(y)$, then $x \equiv y \mod a$ and $x \equiv y \mod b$, thus $x \equiv y \mod ab$ and $x, y$ are equal in $Z_{ab}$.

Vice versa, the conditions $x = x_1 \mod a$ and $x = x_2 \mod b$ specify a unique solution $x \in Z_{ab}$.

This proves, that we can construct a bijection between $Z_{ab}$ and $Z_a \times Z_b$. Thus $|Z_{ab}| = |Z_a||Z_b|$ and $\Phi(ab) = \Phi(a)\Phi(b)$.

\end{document}