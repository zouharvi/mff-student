% !TeX spellcheck = cs_CZ

\documentclass[a4paper]{article}
\usepackage[english]{babel}
\usepackage[utf8x]{inputenc}
\usepackage[T1]{fontenc}
\usepackage{listings}
\usepackage[a4paper,margin=2cm]{geometry}
\usepackage{amsmath}
\usepackage{graphicx}
\usepackage[colorlinks=true, allcolors=blue]{hyperref}
\usepackage{wasysym} % smileys
\usepackage{fancyhdr}
\setlength\parindent{0pt} % indent

% my commands:
\newcommand{\n}{\newline}
\newcommand{\tab}{\hspace{1cm}}

\begin{document}

\thispagestyle{fancy} % beware the difference between \thispagestyle and \pagestyle
\lhead{2nd homework, 15-10-2019}
\rhead{Vilém Zouhar}

\textbf{Theorem: } Non-empty subuniverses (closed under the binary operation) of finite groups by the first definition (associative, cancellative and divisible) are subgroups.

\textbf{Proof: }
Consider a subuniverse $(S,\cdot)$ of finite $(G, \cdot)$.\\

Associativity: $S$ has an associative operator, because $G$ is a group and thus also has an associative operator. \\

Cancellative: Similarly $S$ is cancellative, because $G$ is a group:

\begin{align*}
	&\forall x, y, k \in G: k\cdot x = k\cdot y \rightarrow x = y \\
	&\Rightarrow \\
	&\forall x, y, k \in S \subseteq G: k\cdot x = k\cdot y \rightarrow x = y
\end{align*}

Divisibility:

We consider $x \in S, n = |S|$. If $x\cdot x \cdot x = x$, then $x$ is a unit and is its own inverse.

We also consider $T = \{x, x^1, \ldots, x^{n+1}\}$. From the pigeon principle it follows, that $\exists i, j: i < j: x^i = x^j$. This can be rewritten into: $x^i \cdot e = x^i \cdot x^{j-i}$. From the cancellative property $e = x^{j-i}$. Further, $x^{j-i-1} \in H$ (either $j-i>1 \rightarrow a^{j-i-1} \in S$ or $j-i=1 \rightarrow x^0 = e \in S$).

But $x\cdot x^{j-i-1} = x^{j-i} = e$, so for all $x\in S$ there exists an inverse and by extent (described in my previous homework) $S$ is divisible.


\end{document}