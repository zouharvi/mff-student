% !TeX spellcheck = cs_CZ

\documentclass[a4paper]{article}
\usepackage[english]{babel}
\usepackage[utf8x]{inputenc}
\usepackage[T1]{fontenc}
\usepackage{listings}
\usepackage[a4paper,margin=2cm]{geometry}
\usepackage{amsmath}
\usepackage{amssymb}
\usepackage{graphicx}
\usepackage[colorlinks=true, allcolors=blue]{hyperref}
\usepackage{wasysym} % smileys
\usepackage{fancyhdr}
\setlength\parindent{0pt} % indent

% my commands:
\newcommand{\n}{\newline}
\newcommand{\tab}{\hspace{1cm}}

\begin{document}

\thispagestyle{fancy} % beware the difference between \thispagestyle and \pagestyle
\lhead{1st homework, 08 - 10 - 2019}
\rhead{Vilém Zouhar}

LD := left divisible $\forall a, b \in G: \exists x: a\cdot x = b$\\
RD := right divisible $\forall a, b \in G: \exists x: x\cdot a = b$\\
LC := left cancellative\\
RC := right cancellative\\

\textbf{Proposition:} A semigroup G is a group iff it has a unit and is LC and LD. In this case LD could be replaced with RD, as we will see in the proof. \\

\textbf{Proof:}

$\Rightarrow$ ($G$ is a group $\rightarrow$ G has a unit and is LC and LD):\\
Since $G$ is a group, then it is also divisible and cancellative, hence is LC and LD. In this direction we only need to prove the existence of a unit.
\begin{align*}
 & \forall g,h \in G: \exists l_g, r_g, l_h, r_h: l_g \cdot g = g \cdot r_g,\ l_h \cdot g = g \cdot r_h \text{ (RD, LD)}\\
 & (g\cdot r_g) \cdot h = g\cdot h = g \cdot (l_h \cdot h) = (g \cdot l_h) \cdot h \ \text(associativity)\\
 & (g\cdot r_g) \cdot h = (g\cdot l_h) \cdot h \Rightarrow g\cdot r_g = g\cdot l_h\ \text(RC) \\
 & g\cdot r_g = g\cdot l_h \Rightarrow r_g = l_h\ \text(LC) \\
 & \text{for } g = h \text{ we denote } l_g = r_g = u_g \\
 & \text{but since } \forall g,h \in G: u_g \cdot g = g = g \cdot u_g,\ u_h \cdot g = g = g \cdot u_h \text{ and } u_g = u_h \\
 & \text{then } u_g = u_h = u \text{ is a unit of } G
\end{align*}

$\Leftarrow$ ($S$ is a semigroup with a unit and is LC and LD $\rightarrow$ it is a group):\\
We only need to prove $S$ is RD and RC for it to be a group.\\

RD:
\begin{align*}
 & \forall x \in S\ \exists x_r \in S: x\cdot x_r = u \text{ (unit of G, LD)} \\
 & \text{let } q = x\cdot x_r \\
 & q\cdot q = (x_r\cdot x)\cdot(x_r\cdot x) = x_r\cdot (x\cdot x_r)\cdot x = x_r\cdot (u \cdot x) = x_r\cdot x = q \text{ (associativity)} \\
 & \text{similarly for q: } \exists q_r\in S: q\cdot q_r = u \text{ (LD)}  \\
 & x_r \cdot x = q = q\cdot u = q \cdot (q \cdot q_r)  = (q \cdot q)\cdot q_r = q \cdot q_r = u
\end{align*}

We've shown, that $\forall x \in S\ \exists x': x'\cdot x = x \cdot x' = u$. This means, we've found the inverse. We can multiply the resulting equation from left by $y$ and get the final form for RD. $\forall y, x \in S\ \exists x': (y\cdot x')\cdot x = y\cdot u = y$, thus $\forall y, x \in S\ \exists t: t\cdot x = y$. \\

\textbf{Note:} If we started from RD, we could prove LD, since we used only the one sided divisibility and the existence of a unit. This means, that for each semigroup with a unit, LD is equivalent to RD. Proof follows easily from the previous one.\\

RC:
\begin{align*}
 & \forall g, a, b \in S, \text{ for which } a\cdot g = b\cdot g: \\
 & \exists g': g\cdot g' = u \text{ (LD)} \\
 & (a\cdot g)\cdot g' = (b\cdot g)\cdot g' \\
 & a\cdot(g\cdot g') = b\cdot (g\cdot g') \text { (associativity)} \\
 & a\cdot u = b \cdot u \\
 & a = b 
\end{align*}
This showed, that $\forall g, a, b \in S: (a\cdot g = b\cdot g) \Rightarrow a = b$, which is the definition of RC.

Since $S$ is also LD and RC, we can conclude, that it is a group. $\square$ \\
    
\end{document}
