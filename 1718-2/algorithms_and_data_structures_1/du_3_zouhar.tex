\documentclass[a4paper]{article}
\usepackage[english]{babel}
\usepackage[utf8x]{inputenc}
\usepackage[T1]{fontenc}
\usepackage{listings}
\usepackage[a4paper,top=2cm,bottom=2cm,left=2cm,right=2cm,marginparwidth=1.75cm]{geometry}
\usepackage{amsmath}
\usepackage{graphicx}
\usepackage[colorinlistoftodos]{todonotes}
\usepackage[colorlinks=true, allcolors=blue]{hyperref}

\begin{document}
\vspace{-4cm}
{\fontfamily{pbk}\fontsize{12}{15}\selectfont \hspace{-0.5cm}\text{3. domácí úkol | Vilém Zouhar}}

\section{}
\subsection{Popis}
Pro každý list platí, že je vždy výhodnější vložit policajta do jeho rodiče než do samotného listu (rodič tak pokryje předchozí hranu + nějaké navíc). Tedy (skoro) nikdy nechceme dávat policajta do listů. Ale když ho tam nedáme, tak už nutně musíme dát policajta do rodiče. Tedy vezmeme graf $G$, odstraníme z něj všechny listy, vznikne graf $G'$, ve kterém do listů už policajty musíme dát. Utrhneme všechny listy a vytvoříme graf $G''$, ve kterém jsou všechny listy hlídané, tedy rovnou jdeme k $G''''$, kde opět do listů musíme vložit policajty, atd..

Problém nastává, že nemáme v pěkné složitosti po ruce seznam aktuálních listů. Tedy skoro. Můžeme použít topologické očíslování grafu (algoritmus rozebraný na přednášce), ze kterého dostaneme takový pěkný seznam listů. Tento seznam projdeme a rozmístíme policajty dle popisu.

\subsection{Korektnost}
Rozebráno v popisu. Jdeme od konce, kde jsme si jistí správností umístění.

\subsection{Pseudokód}
\begin{lstlisting}
f = fronta z topologickeho ocislovani (zacina listy)
while f.not_empty():
	p = f.pop()
	// duplikatni
	if p.policeman or p.is_leaf or (for all c in p.children: not c.policeman): 
		p.policeman = true
	else:
		p.parent.policeman = true
\end{lstlisting}
Jednoduše jdeme hladově, ale ve správném pořadí.

\subsection{Složitost}
Topologické očíslování grafu umíme v lineárním čase, pak jen projdeme podle počtu vrcholů, tedy \textit{O(n+m)}.

\subsection{Rozšíření}
Co kdyby vrcholy byly oceněné? Tedy na některé by bylo dražší umístit policajta. Na to by už nefungoval náš přístup, ale museli bychom se uchýlit k rekurzi. U kořene k bychom zavolali \textit{k.policeman = k.price + minimalize(k.left, true) + minimalize(k.right, true) > minimalize(k.left, false) + minimalize(k.right, false);} Museli bychom si však ohlídat, abychom neutekli do exponenciály, tedy za pomocí dynamického programování.

\section{}
\subsection{Popis}
Cesta mezi dvěma nejvzdálenějšími vrchol musí nutně putovat přes kořen (viz. korektnost). Pustíme tedy BFS na kořen, tím nalezneme jeden krajní bod (nejvzdálenější od kořene), který označme A. Z A pak pustíme BFS, čímž najdeme druhý krajní vrchol B. AB jsou pak nejvzdálenější vrcholy, tedy |AB| je průměr.

\subsection{Korektnost}
Co to je vlastně kořen? To by mohl být jakýkoliv vrchol a orientace nás v tomto případě stejně nezajímá. Opravdu si stačí vybrat jakýkoliv vrchol a z něj pustit BFS. Pokud máme jeden krajní bod a pustíme z něj BFS, pak samozřejmě najdeme dva nejvzdálenější body, ale proč BFS z libovolného vrcholu najde jeden krajní bod? Náznak důkazu:

Víme, že dva vrcholy od sebe nejvíce vzdálené jsou oba listy (pokud ne, můžeme prodloužit, spor). Tedy cesta musí jít z A do nejvyššího společného předka A a B a z něj pak do B. Pro všechny možné dvojice A a B platí, že A nebo B je v nejnižší hladině nějak zakořeněného stromu (protože se jedná o průměr). Tedy pokud si ten strom nějak zakořeníme a provedeme BFS z kořene, dostaneme jeden krajní bod.

\subsection{Pseudokód} 
\begin{lstlisting}
BFS(G.random_node, G)
A = node discovered last in BFS
BFS(A, G)
B = node disovered last in BFS
output(|AB|) // taky posledni hladina v poslednim BFS
\end{lstlisting}

\subsection{Složitost}
Děláme pouze dva BFS, tedy potřebujeme \textit{O(n+m)} času.
\end{document}
