\documentclass[a4paper]{article}
\usepackage[english]{babel}
\usepackage[utf8x]{inputenc}
\usepackage[T1]{fontenc}
\usepackage{listings}
\usepackage[a4paper,top=2cm,bottom=2cm,left=2cm,right=2cm,marginparwidth=1.75cm]{geometry}
\usepackage{amsmath}
\usepackage{graphicx}
\usepackage[colorinlistoftodos]{todonotes}
\usepackage[colorlinks=true, allcolors=blue]{hyperref}
\usepackage{wasysym} % smileys
\setlength\parindent{0pt} % indent

% my commands:
\newcommand{\n}{\newline}
\newcommand{\tab}{\hspace{1cm}}

\begin{document}
\text{}\vspace{-0.1cm}
{\fontfamily{pbk}\fontsize{12}{15}\selectfont \hspace{-0.5cm}\text{5. domácí úkol | Vilém Zouhar}}

\section{}
\subsection{Popis}
Budeme vždy hladově přidávat maximální hranu až dokud nepropojíme dva chtěné vrcholy. Některé hrany však přeskončíme, konkrétně ty mezi vrcholy, které jsme už někde potkali a jsou součástí našeho stromu. Ve stromě jsou už cesty jednoznačné, tedy je stačí projít BFS a tuto cestu najít. Je to vlastně Jarníkův algoritmus.
\subsection{Korektnost}
Pro korektnost se stačí jen zamyslet, které hrany opomeneme a musíme odargumentovat, že ty nemůžou být součástí správného řešení. \begin{itemize}
	\item Hrany, které jsou mezi již objevenými vrcholy by v našem stromě vytvořily cyklus (maximalita bez kružnic stromu), ale ta druhá strana cyklu musí nutně mít menší minimální hranu, tedy původní cesta nám stačí.
	\item Hrany, které jsou menší než všechny v našem stromě. Pokud existuje cesta v našem stromě, pak si rozhodně nepomůžeme, když přidáme nějaké menší hrany.
\end{itemize}
\subsection{Pseudokód}
\begin{lstlisting}[language=Python]
tree = jarnik(G)
output( BFS(u, v, tree) )
\end{lstlisting}
\subsection{Složitost}
Děláme BFS a Jarníkův algoritmus. BFS trvá lineárně, ale Jarníkův algoritmus za pomocí matice sousednosti $O(n^2)$, za pomocí binární haldy $O((m+n)\cdot \log(n))$ a za pomocí fibonacciho haldy $O(m+n\cdot \log(n))$, celkově tedy $O(m+n\cdot \log(n))$. Neumím ukázat, že to nelze vyřešit lineárně, ale určitě nejde, protože je to pravda.

\section{}
\subsection{Popis}
Každý cyklus $(a,b,c,d)$ délky $4$ můžeme rozbít na sjednocení dvou cest $(a, c)$ a $(c, a)$. Stačí tedy pro každou dvojici $\{a,c\}$ najít vhodné $\{c ,d\}$, aby $w(a,b)+w(b,c)+w(c,d)+w(d,a)$ bylo co nejmenší. Cesty $(a, c)$ a $(c, a)$ můžeme ale hledat samostatně.
\subsection{Korektnost}
Pro každou dvojici zkoušíme všechna doplnění na čtyřcyklus, není co pokazit.
\subsection{Pseudokód}
\begin{lstlisting}[language=Python]
t_min = 0
res = (/, /, /, /)
for vertex a in V:
    for vertex c in V:
       min_ac = infty
       min_ca = infty
       b, d = /, /
       for vertex x in V:
           if w(a,x) + w(x, c) < min_ac:
               min_ac = w(a,x) + w(x, c)
               b = x
           if w(c,x) + w(x, a) < min_ca:
               min_ca = w(c,x) + w(x, a)
               d = x
       if w(a,b) + w(b, c)+ w(c, d)+ w(d, a) < t_min:
           t_min = w(a,b) + w(b, c)+ w(c, d)+ w(d, a)
           res = (a, b, c, d)
output(res)   
\end{lstlisting}
\subsection{Složitost}
Děláme tři vnořené for cykly délky $n$. Poslední for cyklus by šel vylepšit, pokud bychom procházeli jen sousedy $a$, nebo $b$, ale v úplném grafu bychom dopadli stejně lineárně s $n$, tedy celková složitost algoritmu je $O(n^3)$.
\end{document}
