\documentclass[a4paper]{article}
\usepackage[english]{babel}
\usepackage[utf8x]{inputenc}
\usepackage[T1]{fontenc}
\usepackage{listings}
\usepackage[a4paper,top=1cm,bottom=1cm,left=2cm,right=2cm,marginparwidth=1.75cm]{geometry}
\usepackage{amsmath}
\usepackage{graphicx}
\usepackage[colorinlistoftodos]{todonotes}
\usepackage[colorlinks=true, allcolors=blue]{hyperref}
\usepackage{wasysym} % smileys
\setlength\parindent{0pt} % indent

% my commands:
\newcommand{\n}{\newline}
\newcommand{\tab}{\hspace{1cm}}

\begin{document}
\text{}\vspace{-0.1cm}
{\fontfamily{pbk}\fontsize{12}{15}\selectfont \hspace{-0.5cm}\text{7. domácí úkol | Vilém Zouhar}}

\section{}
\subsection{Popis}
Pokud leží nějaký vrchol na cyklu, tak pak jeho vzdálenost od sebe sama (po puštění Floyd Warshalla) je záporná. Pokud \textit{detekovat} znamená rozhodnout, zdali v grafu záporný cyklus je, nebo není, pak stačí pustit Floyd Warshalla a projít vzdálenosti vrcholů od sebe samých.
\subsection{Korektnost}
Koreknost Floyd Warshalla rozebrána na přednášce. Úprava zjevná.
\subsection{Pseudokód}
\begin{lstlisting}[language=Python]
d = Floyd-Warshall(G)
for i in range(n):
	if d[i,i] < 0:
		return true
return false
\end{lstlisting}
\subsection{Složitost}
Floyd Warshall trvá $O(n^3)$ časově a $O(n^2)$ prostorově. Naše úpravy na tom nic nemění.

\section{}
\subsection{Popis}
Na konci Bellman Forda se pokusíme udělat relax na všechny možné hrany. Pokud někde podmínka relaxu vyjde, je zjevné, že se tam nachází záporný cyklus s hranou, kterou jsme právě zpracovali. Bellman Ford si při relaxu ukládat ke každému vrcholu i jeho předchůdce. Tedy pokud budeme brát předchůdce z vrcholu, u kterého se nám podařilo relaxovat, tak dostaneme kýžený záporný cyklus. Musíme si akorát pamatovat, které vrcholy jsme už našly a pak vzít jen ty, které tvoří cyklus. Celá cesta totiž rozhodně cyklus být nemusí.
\subsection{Korektnost}
Koreknost Bellman Forda rozebrána na přednášce. Předchůdce každého vrcholu určuje jeho nejkratší cestu k $s$. Pokud existuje v grafu záporný cyklus, pak se k $s$ dostaneme nejlépe právě přes něj.
\subsection{Pseudokód}
Možná $\pm 1$ chyby..
\begin{lstlisting}[language=Python]
Bellman-Ford(G, s)
a = null
for each edge (u, v) with weight w in edges:
	if d[u] + w < d[v]:
		a = u
		break
if a == null:
	return "Negative cycle not found"
else:
	arr = [0]*n // indexing by verticies
	i = 1
	for j in range(m):
		if arr[a] != 0:
			return arr, i // verticies with their number higher than i
					 are part of a negative cycle
		arr[a] = i++
\end{lstlisting}
\subsection{Složitost}
Složitost Bellman Forda je  $O(mn)$, naše úprava rozhodně neudělá řádově víc operací než kolik je počet hran.

\end{document}
