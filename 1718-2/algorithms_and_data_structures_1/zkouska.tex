\documentclass[a4paper]{article}
\usepackage[english]{babel}
\usepackage[utf8x]{inputenc}
\usepackage[T1]{fontenc}
\usepackage{listings, graphicx, multicol}
\usepackage[a4paper,top=2cm,bottom=2cm,left=2cm,right=2cm,marginparwidth=1.75cm]{geometry}
\usepackage{amsmath, amssymb, amsfonts, amsthm}
\usepackage[colorinlistoftodos]{todonotes}
\usepackage[colorlinks=true, allcolors=blue]{hyperref}
\usepackage{wasysym} % smileys
\setlength\parindent{0pt} % indent

% my commands:
\newcommand{\n}{\newline}
\newcommand{\tab}{\hspace{1cm}}
\newcommand{\al}[2]{\begin{algorithm}{#1} #2 \end{algorithm}}
\newcommand{\de}[2]{\begin{definition}{#1} #2 \end{definition}}
\newcommand{\te}[2]{\begin{theorem}{#1} #2 \end{theorem}}

% \newtheorem{definition}{Definition}[section]
\theoremstyle{definition}
\newtheorem{definition}{Definition}
\newtheorem{theorem}{Theorem}
\newtheorem{note}{Note}
\newtheorem{algorithm}{Algorithm}

\begin{document}
\text{}\vspace{-0.1cm}
{\fontfamily{pbk}\fontsize{12}{15}\selectfont \hspace{-0.5cm}\text{ADS 1 | Vilém Zouhar}}
\vspace{1cm}

\al{Eucleidův algoritmus}{}
\de{Asymptotická notace}{}
\de{Fronta, zásobník, spojový seznam}{}
\al{BFS, DFS}{}
\de{Halda}{}
\de{Reprezentace grafu (matice sousednosti a vzdálenosti, seznam hran, matice incidence, seznam následníků/sousedů )}{}
\te{Vlastnost DFS}{Stromové hrany tvoří orientovaný les}
\de{Topologické očíslování}{}
\te{Cyklus v DFS}{}
\al{Topologické očíslování v DFS}{}
\al{Nalezení silně souvislých komponent}{}
\te{Vlastnosti algoritmů kritických cest}{TODO}
\al{DAG}{}
\al{Dijkstra}{Korektnost: Sporem přesuneme vrchol z $Q$ do $S$ s tím, že $\delta (s,u) \ne d(u)$ }
\al{Bellman-Ford}{}
\al{APSP násobení matic}{}
\al{Floyd-Warshal}{}
\de{Lehká a bezpečná hrana}{}
\te{Lehká hrana je bezpečná}{}
\al{Jarník}{}
\al{Borůvka}{}
\al{Kruskal}{}
\de{Binární vyhledávací strom}{}
\de{Dokonale vyvážený strom}{}
\de{Červenočerný strom}{}
\de{Výška uzlu, černá výška uzlu}{}
\te{Podstrom obsahuje minimálně $2^{bh(x)}$ listů}{}
\te{Výška nejvýše $2\log_2(n+1)$}{}
\te{Delete a insert v ČČ stromu}{}
\de{AVL strom}{}
\de{B strom}{}
\te{Master theorem}{}
\te{Lemma pro master theorem}{}
\al{Strassenův algoritmus násobení čtvercových matic}{}
\al{Quickselect}{}
\al{Blum - medián lineárně}{}
\al{Vnitřní třídění}{}
\al{Union-find}{}
\al{Hashování}{}
\te{Průměrný čas úspěšného a neúspěšného vyhledávání při řetězení}{}
\de{Univerzální množina hashovacích funkcí}{}
\te{Očekávaný počet kolizí na univerzální množině}{}
\te{Konstrukce univerzální množiny hashovacích funkcí}{}
\te{Řešení kolizí}{}
\te{Průměrný čas úspěšného a neúspěšného vyhledávání při otevřeném adresování}{}
\te{}{}
\end{document}
