\documentclass[a4paper]{article}
\usepackage[english]{babel}
\usepackage[utf8x]{inputenc}
\usepackage[T1]{fontenc}
\usepackage{listings}
\usepackage[a4paper,top=2cm,bottom=2cm,left=2cm,right=2cm,marginparwidth=1.75cm]{geometry}
\usepackage{amsmath}
\usepackage{graphicx}
\usepackage[colorinlistoftodos]{todonotes}
\usepackage[colorlinks=true, allcolors=blue]{hyperref}
\usepackage{wasysym} % smileys
\setlength\parindent{0pt} % indent

% my commands:
\newcommand{\n}{\newline}
\newcommand{\tab}{\hspace{1cm}}

\begin{document}
\text{}\vspace{-0.1cm}
{\fontfamily{pbk}\fontsize{12}{15}\selectfont \hspace{-0.5cm}\text{9. domácí úkol | Vilém Zouhar}}

\section{}
\begin{itemize}
	\item $log_b(a) = log_2(4) = 2; f(n) \in O(n^{2 - \epsilon}) \Rightarrow T(n) \in \Theta(n^2)$
	\item $log_b(a) = log_3(3) = 1; f(n) \in \Theta(n^1\log^0(n)) \Rightarrow T(n) \in \Theta(n \cdot \log(n))$
	\item $log_b(a) = log_4(16) = 2; f(n) \in \Sigma(n^{2 + \epsilon}) \Rightarrow T(n) \in \Theta(n!)$
\end{itemize}

\section{}
\subsection{Popis}
Budeme různě skákat po poli a využívat toho, že Mirek změnil odebráním prvku paritu. Pokud se podívám na jakékoliv sudé místo v poli, tak stojím na začátku dvojice právě tehdy, pokud v dosavadní posloupnosti Mirek ještě nic neukradl. V tomto případě nás zajímá problém jen v poli $a[index, ]$, opačně v $a[, index]$

\subsection{Korektnost}
Rozebrána v popisu.

\subsection{Pseudokód}
\begin{lstlisting}[language=Python]Mi
def solve(i, j):
	index = (j-i)/2
	
	if s[index-1] != s[index] and s[index-1] != s[index+1]:
		output(s[index])
		exit()
	
	if index % 2 == 0:
		s[index] == s[index+1]?solve(index+2, j):solve(i, index-2)
	else:
		solve(i, index)
\end{lstlisting}

\subsection{Složitost}
Paměťove se můžeme zanořit nejvíce v logaritmu a stejně by to šlo přepsat do iterativní formy bez zásobníku. V každém kroku dělíme problém na jeden poloviční velikosti a zbytková funkce je konstantní. To odpovídá $\Theta(n^{\log_2(1)} \cdot \log^0(n)$, tedy celkově $O(\log(n))$.

\section{}
\subsection{Popis}
Využijeme pozorování, že největší obdelník vždy obsahuje alespoň jeden sloupeček. Budeme tedy procházet všechny sloupečky zleva doprava a budeme si rozmýšlet všechny jejich varianty. Začátky obdelníků budou v zásobníku a jejich pravé konce budeme procházet. Šířku obdelníku umíme spočítat v konstantě. Když načteme další sloupeček, tak budeme odebírat ze zásobníku, dokud vršek zásobníku není menší nebo roven. Při odebírání také budeme vždy kontrolovat, zdali jsme nenašli něco lepšího.

\subsection{Korektnost}
Opomíjíme pouze obdelníčky, které jsou obsahově menší, ostatní procházíme všechny.


\subsection{Pseudokód}
\begin{lstlisting}[language=Python]
b = 0
for i in range(n):
	while gram[stack.top()] > gram[i]:
		b = best(b, stack.top()*(i-stack.pop()))
	if gram[i] > gram[i-1]:
		stack.push(i)
		
while s.not_empty():
		b = best(b, stack.top()*(n-stack.pop()))
output(b)
\end{lstlisting}

\subsection{Složitost}
Každý obdelník zpracujeme právě jednou, tedy $\Theta(n)$.

\end{document}
