\documentclass[a4paper]{article}
\usepackage[english]{babel}
\usepackage[utf8x]{inputenc}
\usepackage[T1]{fontenc}
\usepackage{listings}
\usepackage[a4paper,margin=2cm]{geometry}
\usepackage{amsmath}
\usepackage{graphicx}
\usepackage[colorinlistoftodos]{todonotes}
\usepackage[colorlinks=true, allcolors=blue]{hyperref}
\usepackage{wasysym} % smileys
\usepackage{fancyhdr} % styling, headers and footers
\setlength\parindent{0pt} % indent

% my commands:
\newcommand{\n}{\newline}
\newcommand{\tab}{\hspace{1cm}}

\begin{document}

\renewcommand{\headrulewidth}{0pt} % removes horizontal bars from headers and footers
\thispagestyle{fancy} % beware the difference between \thispagestyle and \pagestyle
%\lhead{3. domácí úkol}
\rhead{Vilém Zouhar}
Přehledový seznam vět ke zkoušece z Lineární Algebry II na informatice 2018

\newcommand{\itemv}{\item $\tau$ \hspace{0.2cm}}
\newcommand{\itemd}{\item $\delta$ \hspace{0.2cm}}
\section{Skalární součin}
\begin{itemize}
    \itemd Definice skalárního součinu, standardního skalárního součinu, normy, normy indukované skalárním součinem, metriky
    \itemd Definice kolmosti, ortogonálního a ortonormálního systému vektorů
    \itemv Pythagorova věta, trojúhelníková nerovnost, Cauchy Schwarzova nerovnost, rovnoběžníkové pravidlo
    \itemv Ortonormální systém je lineráně nezávislý, furierovy koeficienty
    \itemv Gram-Schmidtova ortogonalizace, všude ortonormální báze + rozšíření
    \itemv Besselova nerovnost, Parsevalova rovnost
\end{itemize}

\section{Ortogonální doplněk, Ortogonální projekce, Ortogonální matice}
\begin{itemize}
    \itemd Definice ortogonálního doplňku
    \itemv Dva seznamy vlastností ortogonálního doplňku
    \itemd Definice ortogonální projekce
    \itemv Výpočet dané projekce, ortogonální doplněk řádkového prostoru je kernel + důsledky
    \itemv Projekce do sloupcového prostoru, ortogonální projekce do doplňku, metoda nejmenších čtverců
    \itemd Definice ortogonální a unitární matice
    \itemv Vlastnosti ortogonálních matic, součin ortogonálních matic, ortogonální matice a součin
    \itemv Ortogonální matice a lineární zobrazení, stačí 2
\end{itemize}

\section{Determinanty}
\begin{itemize}
    \itemd $sgn$ permutace, definice determinantu, 
    \itemv Řádková linearita determinantu, determinant a elementární úpravy, determinat a singularita
    \itemv Determinant součinu, determinant $A^{-1}$, Laplaceův rozvoj, Cramerovo pravidlo
    \itemd Adjungovaná matice
    \itemv $A\cdot adj(A)$, $A^{-1}$ přes adjungovanou matici, celočíselné hodnoty
    \itemv Objem rovnoběžnostěnu
\end{itemize}

\section{Vlastní čísla}
\begin{itemize}
    \itemd Definice vlastního čísla a příslušeného vlastního vektoru, charakteristický polynom, geometrická násobnost
    \itemv Ekvivalentní definice (charakterizace), vlastní čísla a charakteristický polynom
    \itemd Stopa, spektrum, spektrální poloměr
    \itemv Vlastní čísla, determinant a stopa, vlastní čísla a matice, vlastní číslo komplexně sdružené
    \itemd Definice matice spole4nice
    \itemv Chrakteristický polynom matice společnice
    \itemv Cayley-Hammilton $\star$, důsledek Cayley-Hammilton $\star$
    \itemd Definice podobnosti, diagonalizovatelnost
    \itemv Podobné matice a vlastní čísla
    \itemv Charakterizace diagonalizovatelnosti, lineárně nezávislé vlastní vektory
    \itemv Vlastní čísla $AB$ a $BA$
    \itemd Jordanova buňka, Jordanova normální forma, hermitovská transpozice
    \itemv Podobnost Jordanově normální formě, vlastní čísla hermitovských matic
    \itemv Spektrální rozklad symetrických matic
    \itemv Courant-Fischer o uspořádání vlastních čísel, Perronova, Gerschgorinovy disky
    \itemv Mocninná metoda, Deflace vlastního čísla
\end{itemize}

\section{Pozitivní definitnost}
\begin{itemize}
    \itemd Definice pozitivní definitnosti
    \itemv Základní vlastnosti (linearita, inverze), charakterizace pozitivně definitních matic
    \itemv Rekurence pro pozitivní definitnost, Choleského rozklad, Gaussova eliminace a pozitivní definitnost
    \itemv Sylvestrovo kritérium, skalární součin a pozitivní definitnost, odmocnina z matie
\end{itemize}

\section{Bilineární a kvadratické formy}
\begin{itemize}
    \itemd Definice bilineárních a kvadratických forem, definice příslušné matice
    \itemv Maticové vyjádření, dva důsledky
    \itemv Matice kvadratické formy při změně báze, Sylvestrův zákon setrvačnosti, dva důsledky
    \itemd Signature kvadratické formy, matice Householderovy transformace
    \itemv Householderova transformace
\end{itemize}

\section{Rozklady}
\begin{itemize}
    \itemv QR rozklad
    \itemv SVD rozklad
\end{itemize}

\renewcommand{\itemv}{} \renewcommand{\itemd}{} % don't know how to undefine this, lol

\end{document}