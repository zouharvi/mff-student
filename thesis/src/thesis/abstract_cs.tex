%%% A template for a simple PDF/A file like a stand-alone abstract of the thesis.

\documentclass[12pt]{report}

\usepackage[a4paper, hmargin=1in, vmargin=1in]{geometry}
\usepackage[a-2u]{pdfx}
\usepackage[utf8]{inputenc}
\usepackage[T1]{fontenc}
\usepackage{lmodern}
\usepackage{textcomp}

\begin{document}

%% Do not forget to edit abstract.xmpdata.

Často se uživatelé Internetu potýkají s problémem psaní textu v jazyce, který plně neovládají, a mají tak problém s verifikací kvality překladu. Tomtuto problému říkáme ``outbound translation'' a zkoumáme jej za pomocí nového open-source modulárního systému Ptakopět. Jeho hlavním cílem je zkoumat lidskou interakci se systémy strojového překladu, které jsou posíleny o další subsystémy, jako například zpětný překlad a automatický odhad kvality strojového překladu. Na tento systém navazujeme experimentem na (českých) lidských anotátorech, kteří mají za úkol vyprodukovat otázky v jazyce, kterému nerozumí (němčina), za pomocí Ptakopětu. Zaměřujeme se na tři případy použití v reálném světa (komunikace s IT podporou, popis administrativních problémů a dotazování se na encyklopedické otázky) ze kterých získáváme vhled do různých strategií, které uživatelé používají, když čelí outbound translation. Je známo, že zpětný překlad je nespolehlivý při hodnocení systémů strojového překladu, ovšem náš experiment dokazuje, že zpětný překlad funguje velmi dobře pro běžné uživatele, alespoň u systémů strojvého překladu střední kvality.

\end{document}