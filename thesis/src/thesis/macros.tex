%%% This file contains definitions of various useful macros and environments %%%
%%% Please add more macros here instead of cluttering other files with them. %%%

%%% Minor tweaks of style

% These macros employ a little dirty trick to convince LaTeX to typeset
% chapter headings sanely, without lots of empty space above them.
% Feel free to ignore.
\makeatletter
\def\@makechapterhead#1{
  {\parindent \z@ \raggedright \normalfont
   \Huge\bfseries \thechapter. #1
   \par\nobreak
   \vskip 20\p@
}}
\def\@makeschapterhead#1{
  {\parindent \z@ \raggedright \normalfont
   \Huge\bfseries #1
   \par\nobreak
   \vskip 20\p@
}}
\makeatother

% This macro defines a chapter, which is not numbered, but is included
% in the table of contents.
\def\chapwithtoc#1{
\chapter*{#1}
\addcontentsline{toc}{chapter}{#1}
}

% Draw black "slugs" whenever a line overflows, so that we can spot it easily.
\overfullrule=1mm

%%% An environment for typesetting of program code and input/output
%%% of programs. (Requires the fancyvrb package -- fancy verbatim.)

\DefineVerbatimEnvironment{code}{Verbatim}{fontsize=\small, frame=single}



%%%
%%% Custom commands and notes
%%%

\definecolor{rubine}{rgb}{0.7, 0.1, 0.1}
\newcommand{\VVV}[1]{%
    \noindent%
    \textcolor{rubine}{VVV #1}%
}

\newcommand{\XXX}[1]{%
    \noindent%
    \textcolor{red}{XXX #1}%
}


\newcommand{\backendname}[1]{\textbf{#1}}

\newcommand{\footnotehref}[2]{%
\footnote{\href{#1}{#2}}%
}

% Hack for overfull paragraphs
\tolerance=10000
\hbadness=10000

% Figure from the img/ folder with a caption
\newcommand{\figcap}[3][1]{%
    \begin{figure}[ht]
        \centering
        \includegraphics[width=#1\textwidth]{img/#2}
        \caption{#3}
        \label{fig:#2}
    \end{figure}
}

% Multiple images per one figure
\usepackage{subcaption}

% Force figure position
\usepackage{float}

% Force footnote to stick to the bottom
\usepackage[bottom]{footmisc}

% Used for less spacing in enumerate and itemize
\usepackage{enumitem}
\setlist{noitemsep}


% multirow
\usepackage{multirow}

% Code listing environment
\usepackage{listings}
\definecolor{lightgray}{rgb}{.9,.9,.9}
\definecolor{darkgray}{rgb}{.4,.4,.4}
\definecolor{purple}{rgb}{0.65, 0.12, 0.82}
\lstdefinelanguage{TypeScript}{
  keywords={break, case, catch, continue, debugger, default, delete, do, else, false, finally, for, function, if, in, instanceof, new, null, return, switch, this, throw, true, try, typeof, var, let, string, void, while, with},
  morecomment=[l]{//},
  morecomment=[s]{/*}{*/},
  morestring=[b]',
  morestring=[b]",
  ndkeywords={class, export, boolean, throw, implements, import, this},
  keywordstyle=\color{blue}\bfseries,
  ndkeywordstyle=\color{darkgray}\bfseries,
  identifierstyle=\color{black},
  commentstyle=\color{purple}\ttfamily,
  stringstyle=\color{red}\ttfamily,
  sensitive=true
}
\lstset{
  basicstyle=\ttfamily\small,
  captionpos=b,
  language=TypeScript,
}
\lstdefinelanguage{none}{
  identifierstyle=
}
% This renames the lstlistings title
\renewcommand*{\lstlistlistingname}{List of Listings}
\renewcommand*{\lstlistingname}{List of Listings}


% 1cm hspace gap by default
\newcommand{\ttab}[1][1]{%
    \text{} \hspace{#1 cm}%
}
% cref
\usepackage[noabbrev,capitalize]{cleveref}

% Ptakopět 
\def\ptakopet{Ptakopět}
