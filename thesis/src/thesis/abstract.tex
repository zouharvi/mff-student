%%% A template for a simple PDF/A file like a stand-alone abstract of the thesis.

\documentclass[12pt]{report}

\usepackage[a4paper, hmargin=1in, vmargin=1in]{geometry}
\usepackage[a-2u]{pdfx}
\usepackage[utf8]{inputenc}
\usepackage[T1]{fontenc}
\usepackage{lmodern}
\usepackage{textcomp}

\begin{document}

%% Do not forget to edit abstract.xmpdata.

It is not uncommon for Internet users to have to produce text in a foreign language they have very little knowledge of and are unable to verify the translation quality. We call the task ``outbound translation'' and explore it by introducing an open-source modular system Ptakopět. Its main purpose is to inspect human interaction with machine translation systems enhanced by additional subsystems, such as backward translation and quality estimation. We follow up with an experiment on (Czech) human annotators tasked to produce questions in a language they do not speak (German), with the help of Ptakopět. We focus on three real-world use cases (communication with IT support, describing administrative issues and asking encyclopedic questions) from which we gain insight into different strategies users take when faced with outbound translation tasks. Round trip translation is known to be unreliable for evaluating MT systems but our experimental evaluation documents that it works very well for users, at least on MT systems of mid-range quality.

\end{document}