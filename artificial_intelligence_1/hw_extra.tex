% !TeX spellcheck = cs_CZ

\documentclass[a4paper]{article}
\usepackage[english]{babel}
\usepackage[utf8x]{inputenc}
\usepackage[T1]{fontenc}
\usepackage{listings}
\usepackage[a4paper,margin=2cm]{geometry}
\usepackage{amsmath}
\usepackage{amssymb}
\usepackage{graphicx}
\usepackage[colorlinks=true, allcolors=blue]{hyperref}
\usepackage{wasysym} % smileys
\usepackage{fancyhdr}
\setlength\parindent{0pt} % indent

% my commands:
\newcommand{\n}{\newline}
\newcommand{\tab}{\hspace{1cm}}

\begin{document}

\thispagestyle{fancy} 
\lhead{HW extra, 05-01-2020}
\rhead{Vilém Zouhar}

\section{Vlastnosti heuristik}
\subsection{Max z heuristik}
$p$ je libovolný následník vrcholu $n$. Předpokládáme $\forall i: H_i$ je konzistentní.
\begin{align*}
H'(n) = \max(H_1(n), H_2(n), \ldots, H_N(n)) &\le (H_1(p) + c(n,p), H_2(p) + c(n,p), \ldots, H_N(p) + c(n,p)) \\
&= max(H_1(p), H_2(p), \ldots, H_N(p)) + c(n,p) = H'(p) + c(n,p) \\
\Rightarrow H'(n) \le H'(p) + c(n,p)
\end{align*}

\subsection{Přípustná heuristika}
Tvrzení není pravdivé. Uvažme lineární cestu délky 3 (4 vrcholy $a, b, c, d$, $d$ je cílový). Každá hrana má cenu $1$.

\begin{tabular}{|l|c|c|c|c|}
\hline
    Vrcholy:& a& b & c & d \\
    Opravdová vzdálenost:& 3 & 2 & 1 & 0 \\
    Heuristika:& 3 & 1 & 1 & 0 \\
\hline
\end{tabular}

\vspace{0.5cm}

Heuristika je jistě přípustná, neboť dává vždy nejvýše takový odhad, jako je opravdová vzdálenost. Není však konzistentní, neboť $h(a) \nleq h(b) + c(a,b) = 1 + 1 = 2$

\subsection{Součet heuristik}
Tvrzení není pravdivé. Uvažme podobný případ cesty, ovšem se skoro konstantní (přípustnou i konzistentní) heuristikou $H_K$.

\begin{tabular}{|l|c|c|c|c|}
\hline
    Vrcholy:& a& b & c & d \\
    Opravdová vzdálenost:& 3 & 2 & 1 & 0 \\
    $H_K$:& 1 & 1 & 1 & 0 \\
    $H' = H_K + H_K$:& 2 & 2 & 2 & 0 \\
\hline
\end{tabular}

\vspace{0.5cm}

Pak evidentně $H'$ není přípustná, neboť $2 = H'(c) > f(c) = 1$.

\subsection{Dvojnásobná heuristika}

Tvrzení je pravdivé. Předpokládejme, že A* najde horší řešení. V libovolném bodě výpočtu se ve fringe vyskytoval konec prefixu cesty nejvýše dvakrát horší, než optimální. Těsně předtím, než A* došel až do cíle, tak měl na výběr mezi tímto cílem, nebo vrcholem v cestě, která je nejvýše dvakrát horší. Potom si ale nemohl vybrat ten cíl, neboť se vybírá minimum z fringe.

\section{Žáby}

\section{Hraní her}

\section{Logika}

\end{document}
