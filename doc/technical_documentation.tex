\documentclass[11pt]{article}
\usepackage[utf8]{inputenc}
\usepackage[T1]{fontenc}
\usepackage{indentfirst}
\usepackage{textcomp}
\usepackage{tabto}
\usepackage[shortcuts]{extdash}
\usepackage{listings}
\usepackage[a4paper, total={6in, 9in}]{geometry}

\lstset{literate=%
    {á}{{\' a}}1
    {é}{{\' e}}1
    {è}{{\` e}}1
    {í}{{\' i}}1
    {ý}{{\' y}}1
    {ó}{{\' o}}1
    {ú}{{\' u}}1
    {ů}{{\r{u}}}1
    {š}{{\v s}}1
    {č}{{\v c}}1
    {ř}{{\v r}}1
    {ě}{{\v e}}1
    {ž}{{\v z}}1
    {ú}{{\' u}}1
    {ä}{{\" a}}1
    {ö}{{\" o}}1
    {ł}{{\l}}1
    {ę}{{\k e}}1
    {ą}{{\k a}}1
    {ź}{{\" o}}1
    {ś}{{\" s}}1
    {ť}{{\` t}}1
}
\lstset{basicstyle=\ttfamily\footnotesize,breaklines=true}

\title{\textbf{Technická dokumentace LANSHER }}
\author{Vilém Zouhar}
\date{2017}

\usepackage{amsmath}
\begin{document}

\maketitle

\section{Úvod a použité nástroje}
LANSHER slouží k rozlišení jazyka na základě velkého množství trénovacích dat. Je naprogramován v jazyce Python3. Tato verze je programem striktně vyžadována, aby nedošlo k neočekávanému chování. Využívá se běžně dostupných knihoven, konkrétně \textbf{json, argparse, re} (regex) a \textbf{pickle} (serializace).


\section{Fungování programu}
Program je založený na tvorbě \textit{slovníku} ze vstupních definic jazyka. Ze vstupu na rozeznání se vytvoří také \textit{slovník} jazyka. Není tedy rozdíl v porovnávání dvou jazyků a rozeznávání vstupního textu (porovnání se všemi jazyky v databázi).

Termínem \textit{slovník jazyka} je myšleno přiřazení slova \textbf{sl} k počtu výskytu \textbf{freq\textsubscript{sl}}. Pro větu \textit{Alenka tváří tvář Alence vyslovila: "Jmenuji se Alenka"} se jedná o zobrazení:


\begin{center}
\begin{tabular}{ l c l }
 \textit{alenka} & \textrightarrow & \textit{2} \\ 
 \textit{tváří} & \textrightarrow & \textit{1} \\ 
 \textit{tvář} & \textrightarrow & \textit{1} \\ 
 \textit{alence} & \textrightarrow & \textit{1} \\ 
 \textit{vyslovila} & \textrightarrow & \textit{1} \\ 
 \textit{jmenuji} & \textrightarrow & \textit{1} \\ 
 \textit{se} & \textrightarrow & \textit{1} \\ 
 & \hspace{100pt} &  \\
\end{tabular}
\end{center}

Informace o interpunkčních znaménkách a velikosti písmen je pro jednoduchost srovnávání zahozena (vhodným regulérním výrazem). V programu je objekt \textit{slovníku} implementován za pomocí \textit{dictionary}, což má operace přidání prvku, získání prvku průměrně v konstantním čase.

Pro srovnání dvou jazyků, například výše uvedené věty (\textit{S\textsubscript{1}}) a českého překladu \textit{Alenka v říši divů} (\textit{S\textsubscript{cs}}), první získáme množinu všech vzorů vyskytující se v obou \textit{slovnících} (\textit{M}) a první část jejich podobnosti určíme vztahem:

\begin{center}
$ |S_1S_{cs}|^\prime = \sum_{i=1}^{|M|} \ \ log(S_1(M_i)\cdot S_{cs}(M_i) +1) $
\end{center}

Je to tedy logaritmus vzájemného násobku četnosti dvou klíčů, pro všechny klíče vyskytující se v obou množinách. Člen $+1$ byl zvolen kvůli tomu, že v případě $S_1(m) = S_{cs}(m) = 1$ je funkční hodnota logaritmu 0, tedy chceme pozitivně ocenit i jeden společný výskyt.

Četnosti se násobí, aby se získala větší podobnost při větší míře výskytu. Logaritmus je použití proto, aby se odfiltrovaly patologické případy. Jeden je demonstrován na dalším případě:

S\textsubscript{en} (slovo \textit{to} je v angličtině zastoupeno hojně):
\begin{center}
\begin{tabular}{ l c l }
 \textit{to} & \textrightarrow & \textit{31} \\ 
 \textit{..} & \textrightarrow & \textit{..} \\ 
 \textit{..} & \textrightarrow & \textit{..} \\ 
 \hspace{30pt} & \hspace{100pt} & \hspace{30pt} \\
\end{tabular}
\end{center}

S\textsubscript{cs}:
\begin{center}
\begin{tabular}{ l c l }
 \textit{to} & \textrightarrow & \textit{5} \\ 
 \textit{není} & \textrightarrow & \textit{4} \\ 
 \textit{pravda} & \textrightarrow & \textit{2} \\ 
 \textit{..} & \textrightarrow & \textit{..} \\ 
 \textit{..} & \textrightarrow & \textit{..} \\ 
 \hspace{30pt} & \hspace{100pt} & \hspace{30pt} \\
\end{tabular}
\end{center}

S\textsubscript{2} (věta \textit{"To není pravda!"}):
\begin{center}
\begin{tabular}{ l c l }
 \textit{to} & \textrightarrow & \textit{1} \\ 
 \textit{není} & \textrightarrow & \textit{1} \\ 
 \textit{pravda} & \textrightarrow & \textit{1} \\ 
 \hspace{30pt} & \hspace{100pt} & \hspace{30pt} \\
\end{tabular}
\end{center}

Byť má slovník angličtiny (zpracovaný na základě \textit{Alice in Wonderland}) s větou společné jen jedno slovo, má velkou četnost. Český slovník má definované všechny tři slova ve větě, ale ta mají menší četnost. V případě bez logaritmů by se tedy věta nekorektně rozpoznala jako anglická. Využíváme toho, že logaritmus je funkce rostoucí, avšak s klesající derivací~($\frac{1}{x}$).

Další krajní případ může nastat tehdy, když databázi přetrénujeme. Tedy samotný jazyk bude mít definovaná slova, která bychom do něj běžně nezařadili. V trénovacím textu pro češtinu se může vyskytnou věta \textit{"V němčině 'guten Tag' znamená 'dobrý den'"}.  Výslední slovník by mimo jiné obsahoval i slova pro \textit{guten, tag}. Součet podobností tedy vydělíme počtem definovaných slov v jazyce, čímž budeme slovník penalizovat za svoji rozsáhlost.

Podobnost dvou slovníků, $|S_1S_2|$, definujeme jako:

\begin{center}
$|S_1S_2| = \frac{\sum_{i=1}^{|M|} \ \ log(S_1(M_i)\cdot S_{cs}(M_i) +1)}{log(|S_1||S_2|+1)} $
\end{center}

Kromě podobnosti slovníků se k podobnosti jazyků používá (byť v řádově menší míře) i porovnání četnosti znaků, stejným způsobem.

\section{Struktura programu}
Celý program je napsaný v souboru \textit{lansher.py}. K jeho spuštění je zapotřebí Python3. Před vlastním programem je definována řada funkcí, přičemž všechny fungují průměrně v \textit{O(n)} čase.

\begin{itemize}
\item \textit{clean\_data}

Vyčistí vstupní řetězec od interpunkčních znamének a převede ho do lowercase.
\item \textit{element\_frequency}

Vrací objekt typu \textit{dictionary}, kde funkční hodnota každého klíče je jeho četnost ve vstupním poli.
\item \textit{join\_frequencies}

Slouží k sloučení dvou slovníků jednoho jazyka.
\item \textit{create\_lang\_object}

Vytvoří, popřípadě sloučí, slovní a znakové slovníky jazyka. Také vypočítá počet slov ve slovsnících. Odkazuje se na \textit{join\_frequencies}.
\item \textit{add\_to\_database}

Řídící funkce odkazující se na \textit{create\_lang\_object}. Přidává nově vytvořený jazyk do databáze programu.
\item \textit{add\_lang\_file}

Vytvoří objekt jazyka na základě vstupního souboru (JSON). Obsahuje validaci vstupu.
\item \textit{add\_lang\_database}

Vytvoří objekt jazyka na základě vstupního souboru již dřívě použitého a uloženého jazyka. Obsahuje validaci vstupu.
\item \textit{save\_lang\_database}

Uloží aktuální databázi programu do souboru za pomocí serializace \textit{pickle}.
\item \textit{distance\_langs}

Vypočítá vzdálenost dvou vstupních jazyků, je možné parametrizovat \textit{WORD\_CHAR\_RATIO} - tedy poměr váhy četností slov a znaků.
\item \textit{compare\_against\_database}

Porovná vstupní jazyk se zbytkem databáze programu a zobrazí výsledky v procentech.
\item \textit{podmínka na konci programu}

Obsahuje zpracování přepínačů a řídí celý program.
\end{itemize}

Program je ošetřen oproti neplatným vstupům, je možné používat různé přepínače a zejména nápovědu \textbf{-h}.

\section{Reálná funkčnost}
S programem je dodávaná zpracovaná databáze 10 jazyků z různých překladů knihy \textit{Alenka v říši divů}. Změřme tedy dobu zpracovávání databáze a její následné ukládání:
\begin{lstlisting}
time ./lansher.py -ls alice_full.json  -sd alice_full.ld
\end{lstlisting}

Výstup:
\begin{lstlisting}
real	0m0.478s
user	0m0.448s
sys 	0m0.024s
\end{lstlisting}

Tedy půl vteřiny. Vstupní JSON má $\sim$2.3MB, zatímco zpracovaná databáze $\sim$0.8MB Pokud chceme použít ji, je na tom čas mnohem lépe:
\begin{lstlisting}
time ./lansher.py -ld alice_full.ld -i "Byl pozdní večer - první máj.."
\end{lstlisting}

Výstup:
\begin{lstlisting}input text comparison:
cs (59.713%)
es (6.132%)
fr (5.333%)
pl (5.242%)
no (5.124%)
de (4.966%)
en (4.824%)
it (4.803%)
ja (3.715%)
ru (0.149%)

real	0m0.066s
user	0m0.060s
sys 	0m0.008s
\end{lstlisting}

Pokud se nejedná o příliš krátkou větu, která není reprezentativní pro daný jazyk, dokáže program ve většině případů jazyk rozeznat správně.
\end{document}