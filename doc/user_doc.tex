\documentclass[a4paper]{article}
\usepackage[english]{babel}
\usepackage[utf8x]{inputenc}
\usepackage[T1]{fontenc}
\usepackage{listings}
\usepackage[a4paper,top=2cm,bottom=2cm,left=2cm,right=2cm,marginparwidth=1.75cm]{geometry}
\usepackage{amsmath}
\usepackage{graphicx}
\usepackage[colorinlistoftodos]{todonotes}
\usepackage[colorlinks=true, allcolors=blue]{hyperref}
\usepackage{wasysym} % smileys
\setlength\parindent{0pt} % indent

% my commands:
\newcommand{\n}{\newline}
\newcommand{\tab}{\hspace{1cm}}

\begin{document}
\text{}\vspace{1.1cm}
\begin{center}
	\fontfamily{pbk}
	{\huge \text{Grewh2D}} \vspace{0.3cm} \\ 
	{\small \text{Genetically Refined Wheels in 2D}} \vspace{0.1cm} \\
	{\small \text{version 0.1.0}} \vspace{0.5cm} \\
	{\normalsize \text{User documentation}} \vspace{0.2cm} \\
	{\normalsize \text{by Vilém Zouhar}} \vspace{0.5cm} \\
\end{center}

\section*{About}
The purpose of Genetically Refined Wheels in 2D (abbrev. Grewh2D) is to demonstrate the beauty of genetic algorithms. It was created by Vilém Zouhar as a semestral project at Charles University Faculty of Mathematics and Physics in 2018.
Genomes in this program represent polygons with attached circles, also known as cars. Their goal is to reach the other end of the map.

\section*{Getting Grewh2D}
Grewh2D can be run either in WebGL environment, or locally via Windows or Linux executables. WebGL, binaries and latest versions of user and technical documentation are hosted at \href{http://vilix.xyz/s/grewh2d/}{vilix.xyz/s/grewh2d}. Note that Windows and Linux executables don't have to be installed and provide much higher performance and stability. \\

WebGL version has been tested on Chrome 65.0.3325.181 64-bit, Windows, Windows version on Win10 (all updates till april 2018), 64-bit and lastly the Linux version has been tested on Debian 9.4 64-bit.

\section*{Usage}
The program is divided into three parts:
\begin{enumerate}
	\item Population showcase 
	\item Simulation
	\item Statistics
\end{enumerate}
The most common workflow is: $1. \rightarrow 2. \rightarrow 3. \rightarrow 1. \rightarrow ...$
\subsection*{Population showcase}
The whole population can be rerandomized with the \textit{New population} button. \textit{Start} changes the scene to 2. (Simulation).

\subsection*{Simulation}
Simulation provides a method to sort genomes by their fitness. Once all individuals are killed, new population is created based on previous results. The simulation environment provides these options:
\begin{itemize}
	\item{Population size} \hspace{0.92cm} Size of the next population
	\item{Terrain difficulty} \hspace{0.67cm}  Hardness of the terrain (achieved by average height difference).\\ 
	\text{}\hspace{3.39cm}  For new terrain, click \textit{New terrain}
	\item{Gravity} \hspace{2.1cm} Downward gravity force
	\item{Kill timer} \hspace{1.8cm} How fast to kill the least capable individual
	\item{Simulation speed} \hspace{0.71cm} Speed of the physical simulation
	\item{Individual/parents} \hspace{0.39cm} How many genomes should be based on mutating one individual or be derived \\
	\text{}\hspace{3.39cm} from parents
	\item{Mutation} \hspace{1.8cm} Mutation size
\end{itemize}
By default the camera follows the leader. Clicking the \textit{camera button} cycles the camera mode: \textit{second, best, last} and \textit{free} mode. \textit{Mutate now} kills of all individuals by their respective position. \textit{Statistics} changes the scene to 3.

\subsection*{Statistics}
A simple screen displaying two graphs: \textit{average fitness by generation} and \textit{best fitness by generation}.

\section*{Notes}
Please note that genetic algorithms (similar to the one used in Grewh2D) are by definition nondeterministic. Achieving the end is not ensured as randomness is introduced.
\end{document}
