\documentclass[a4paper]{article}
\usepackage[english]{babel}
\usepackage[utf8x]{inputenc}
\usepackage[T1]{fontenc}
\usepackage{listings}
\usepackage[a4paper,margin=2cm]{geometry}
\usepackage{amsmath}
\usepackage{graphicx}
\usepackage[colorinlistoftodos]{todonotes}
\usepackage[colorlinks=true, allcolors=blue]{hyperref}
\usepackage{wasysym} % smileys
\usepackage{fancyhdr}
\setlength\parindent{0pt} % indent

% my commands:
\newcommand{\n}{\newline}
\newcommand{\tab}{\hspace{1cm}}

\begin{document}

\thispagestyle{fancy} % beware the difference between \thispagestyle and \pagestyle
\lhead{7. cvičení (14. 11. 2018)}
\rhead{Vilém Zouhar}

\section{}
Stačí spočítat kovarianci (speciální $E$), směrodatné odchylky a pak vhodit do zlomku.

\section{}
Úplně stejně. Výsledek by měl být však jiný, než v jedničce.

\section{}
\subsection{}
Po integraci vyjde, že $c=4$
\subsection{}
Je třeba vypočítat $var(XY)$, proto $E[XY]$ a $E[X^2Y^2$.
\subsection{}
Jsou. (není to však opačná implikace?)
\subsection{}
Vhodně do matice $4\times 4$ naházíme $var(X), cov(X,Y), cov(Y, X), var(Y)$
\subsection{}
Střední hodnota je přímo lineární. U variace je nutné nezapomenout na všechny kovariace (ty jsou však v tomto případě nulové).

\section{}
Přes integrály: $cov(X,Y) = 0$\\
Výsledný vektor nemá hustotu. Jsou závislé, neboť $P[x \in (-0.5, 0.5)] \cdot P[x^2 \in (0.5, 1)] \ne P[x \in (-0.5, 0.5) \wedge x^2 \in (0.5, 1)]$

\section{}
Integrál kruhu. Rozhodně tedy musí být závislé (všechny nezávislé jsou v kvádru)

\section{}
$Y_i \rightarrow Bi(1, p), X \rightarrow Bi(n, p)$\\
$ EX = E[\sum Y_i] = \sum E[Y_i] = np$ \\
$var(X) = \sum var(Y_i) = \sum var(Y_i) + \sum \sum cov(Y_i, Y_j) = np(1-p) + 0$

\end{document}