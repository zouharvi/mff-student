\documentclass[a4paper]{article}
\usepackage[english]{babel}
\usepackage[utf8x]{inputenc}
\usepackage[T1]{fontenc}
\usepackage{listings}
\usepackage[a4paper,margin=2cm]{geometry}
\usepackage{amsmath}
\usepackage{graphicx}
\usepackage[colorinlistoftodos]{todonotes}
\usepackage[colorlinks=true, allcolors=blue]{hyperref}
\usepackage{wasysym} % smileys
\usepackage{fancyhdr}
\setlength\parindent{0pt} % indent

% my commands:
\newcommand{\n}{\newline}
\newcommand{\tab}{\hspace{1cm}}

\begin{document}

\thispagestyle{fancy} % beware the difference between \thispagestyle and \pagestyle
\lhead{9. cvičení (5. 12. 2018)}
\rhead{Vilém Zouhar}

\section{}
\subsection{}
$ \sim Bi(n,p) $

\subsection{}
Je třeba napočítat ručně (tabulka).

\subsection{}
Pro $n=2$:
\begin{align*}
	& P(X_1 + X_2 = k) = \sum_0^\infty P(X_1 + X_2 = k| X_2 = l)\cdot P(X_2 = l) = \sum_0^k P(X_1 k-l)\cdot P(X_2 = l) = \sum_0^k \frac{\lambda^{k-l}e^{-\lambda}}{(k-l)!} \cdot \frac{\lambda^l e^{-\lambda}}{l!} = \\
	& \frac{e^{-2\lambda}\lambda^k}{k!} \sum_0^k \frac{k!}{(k-l)!l!} = \frac{e^{-2\lambda}\lambda^k}{k!} 2^k = \frac{e^{-2\lambda}(2\lambda)^k}{k!} \rightarrow \sim Po(2\lambda)
\end{align*} 
Obecně $n\lambda$

\section{}
$ P(\sum_0^{100} X_i > 60) = P(\frac{ \sum_0^{100} X_i - 50}{\sqrt{100 \frac{1}{2}\cdot(1-\frac{1}{2})}} > \frac{60 - 50}{5}) = 1-P(\frac{ \sum_0^{100} X_i - 50}{\sqrt{100 \frac{1}{2}\cdot(1-\frac{1}{2})}} < 2) \approx 1-\Phi(2) $

\section{}
\subsection{}
Prostě CLV na vše
\end{document}