\documentclass[a4paper]{article}
\usepackage[english]{babel}
\usepackage[utf8x]{inputenc}
\usepackage[T1]{fontenc}
\usepackage{listings}
\usepackage[a4paper,margin=2cm]{geometry}
\usepackage{amsmath}
\usepackage{graphicx}
\usepackage[colorinlistoftodos]{todonotes}
\usepackage[colorlinks=true, allcolors=blue]{hyperref}
\usepackage{wasysym} % smileys
\usepackage{fancyhdr}
\setlength\parindent{0pt} % indent

% my commands:
\newcommand{\n}{\newline}
\newcommand{\tab}{\hspace{1cm}}
\newcommand{\imwb}[2]{\textbf{#1} - #2 \\}
\newcommand{\imwc}[2]{\textbf{#1} - #2}
\newcommand{\ims}[1]{\textbf{#1}}

\begin{document}
\pagenumbering{arabic}
\thispagestyle{fancy} % beware the difference between \thispagestyle and \pagestyle
\lhead{Úvod do počítačové lingvistiky}
\rhead{Vilém Zouhar}

Stručný zápis témat a krátkých vysvětlivek k předmětu Úvod do počítačové lingvistiky vyučované doc. Kuboněm na MFF v ZS18/19. Zdrojem tohoto textu je samotná přednáška, ale také doplňková skripta prof. Hajičkové (Úvod do teoretické a počítačové lingvistiky - 1. svazek) a materiály poskytované k tomuto předmětu.

% Principy kvantity, kvality, relace a způsobu.
\subsection*{Přirozený jazyk}
Funkce:
\begin{itemize}
\item popis věcí a jevů v reálném světě
\item popis abstraktních struktur
\item komunikace směřující k řešení
\item popis samotného jazyka
\end{itemize}

Zásady:
\begin{itemize}
\item všeobecnost (nejčastější způsob komunikace)
\item využitelnost (je agilní a nestárne)
\item obsah (je úplný, včetně meta popisu)
\item vágnost (jistá neurčitost je základem inovativního myšlení)
\item vícevrstevnost (dokáže popisovat i různé úrovně komunikace)
\item zkratkovitost (bývá kratší než dotazy v umělých jazycích)
\end{itemize}

Víceznačnost:
Občas formální problém, ale i častý základ vtipu, nebo myšlenky.
\pagebreak
\section*{Morfologie}
Počátek už v -400 v popisu sanskrtu. Předmětem morfologie je \ims{studium vnitřní struktury slov}.

\imwb{Lexikologie}{slova jako jednotky slovní zásoby}
\imwb{Lexikografie}{sestavování slovníků}
\imwb{Morfém}{nejmenší znaková jednotka jazyka nesoucí význam (lexikální|gramatický); skládá se ze sémat}
\imwb{Séma}{nejmenší znaková jednotka jazyka vztahující se k formě}
\imwb{Foném}{nejmenší zvuková jednotka}
\begin{center}
Za.hrad.ou \\
Předpona (prefix) . mofrém (lexikální) . morfém (gramatický, skládá se ze tří sémat (3. p., č. j., r. ž.))
\end{center}
\begin{center}
Dom.ům
morfém (lexikální) . morfém (gramatický - dvě sémata (plurál + dativ))
\end{center}

\imwb{Skoňování}{deklinace}
\imwb{Časování}{konjugace}
\imwb{Tvaroslovné dublety}{kolize v odvození z různých slovních základů | žena (1/5), tři (5/4), stát (5/1), už (5/6}
\imwb{Alternace}{změna hlásek uvnitř kmene v tvarování | vůz - vozu, švec - ševce, prkno - prken}
\imwb{Alomorfy}{varianty téhož morfému/kmene | -řík-, -říc-, -řek-}
\imwb{Autosémantická slova}{nesou význam | podstatná, přídavná jména, zájmena, číslovky, slovesa, příslovce, citoslovce (většinou ohebá - první čtyři se skloňují, slovesa časují a některá příslovce stupňuj)}
\imwb{Synsémantická slova}{nemají samy o sobě plnohodnotný význam | obvykle předložky, spojky, částice (všechna neohebná) }

\subsection*{Morfologická typologie}
\begin{itemize}
\item \imwc{analytické (izolační)}{slovo = morfém | angličtina, čínština} 
\item \imwc{syntetické (flektivní, aglutinační)}{slovo > morfém, }  \\
u flektivních jazyků (slovanské) mají afixy více funkcí, zatímco u aglutinačních jazyků (maďarština) je zřejmá korespondence a řetězí se častěji
\item \imwc{polysyntetické ("přehnané aglutinační")}{v případě, kdy třeba sloveso dostane takové množství afixů, že nese význam věty | eskymánština } 
\end{itemize}
	
	
\subsection*{Metody zpracování morfologie}
Založeno na:
\begin{itemize}
\item slovo jako posloupnost morfémů
\item slovo = $P_\alpha \cdots (P_b(P_a($kořen$))\cdots)$, kde $P_i$ je pravidlo
\item je dána tabulka vzorů + všechny jejich tvary; u každého slova určíme vzor a z toho odvodíme gramatické kategorie
\end{itemize}

\subsection*{Dvojúrovňová morfologie}
Místo funkce, která generuje výsledné slovo se provádí dva výpočty zároveň: lexikální a povrchový.
\begin{center}
	\texttt{b a b y + 0 s} \\
	\texttt{b y b i 0 e s} \\
	\texttt{0} obvykle značí prázdné místo (protějšek nemá žádnou realizaci) \\
	spojuje se do zápisu: \texttt{b:b a:y b:b y:i +:0 0:e s:s} \\
	některé stavy ohlašují gramatické kategorie (třeba \texttt{s:s $\rightarrow$ plural})
\end{center}
Poměrně nový přístup. Doposud byl problém ten, že dva různé generovací průběhy mohly skončit ve stejném slově a tudíž zpětná analýza není jednoznačná.

\subsection*{Česká morfologie}
Každé slovo má 13 (+2 rezerva) značek, které mají dle pozice svůj význam a popisují morfologické kvality.
\begin{center}
\textit{nejnezajímavější}\\
\texttt{AAFP3----3N----}
adjective . regular . feminine . plural . dative . - . superlative . negated . - \\
no poss. gender, no poss. number, no person, no tense, no voice, base variant
\end{center} 
\imwc{Lemma}{slovníková reprezentace základního tvaru | lesům, lesy, lesích $\rightarrow$ les} \\
\text{}\hspace{1.5cm} stát/slov. $\rightarrow$ stát-1, šel $\rightarrow$ jít

\subsection*{Morfologická analýza}
\begin{center}
$\rightarrow$ \\
\texttt{Prezident rezignoval na svou funkci.} \\

$\leftarrow$ \\
\end{center}
\texttt{<csts>\\
<f cap>Prezident<MMl>prezident<MMt>NNMS1-----A----\\
<f>rezignoval<MMl>rezignovat\_:T<MMt>VpYS---XR-AA---\\
<f>na<MMl>na<MMt>RR--4----------<MMt>RR--6----------\\ <f>svou<MMl>svůj-1\_\^(přivlast.)<MMt>P8FS4---------1<MMt>P8FS7---------1\\
<f>funkci<MMl>funkce<MMt>NNFS3-----A----<MMt>NNFS4-----A----\\
<MMt>NNFS6-----A----\\
<D>\\
<d>.<MMl>.<MMt>Z:-------------\\
</csts>}

Použití:
\begin{itemize}
\item \imwc{morfologická analýza}{výsledek je seznam lemmat a značek (klidně více dvojic)}
\item \imwc{morfologické značkování}{výběr správné dvojice lemma-značka}
\item \imwc{částečná morfologická desambiguace}{pomocí pravidel jazyka zahazujeme značky, které by v daném postavení nemohly reprezentovat správnou větu}
\item \imwc{lemmatizace}{výběr správného lemmatu, ze kterého byl odvozen vstupní tvar (třeba pro vyhledávání v textu)}
\item \imwc{stemming}{odříznutí koncovky, dostaneme kořen slova}
\item \imwc{generování}{výběr správného slovního tvaru, pokud známe lemma a dostatek gramatických kategorií}
\end{itemize}

\subsection*{Kontrola překlepů}
\begin{enumerate}
	\item nalézt všechny výskyty a opravit je
	\item opravená verze musí sedět do kontextu
	\item neznámá slova nejsou chyby
	\item no false positive
	\item co nejvíce automatická korekce
	\item vše počítat co nejrychleji
\end{enumerate}

Dva triviální přístupy:
\begin{enumerate}
\item porovnání řetězců se slovy ve slovníku 
	\begin{itemize}
	\item seznam všech možných slovních tvarů daného jazyka / seznam lemmat + morfologická analýza
	\item výhoda: spolehlivé a jednoduché
	\item nevýhoda: závislé na kvalitě slovníku, který se musí udržovat; neznámá slova jsou chybná
	\end{itemize}
\item srovnání skupin znaků (dvojice, trojice, .. n-gramy) a hledání nedovolených kombinací
	\begin{itemize}
	\item výhoda: výpočetně rychlé, nezávislé na slovníku
	\item nevýhoda: spousta chybných slov se skládá ze správných kombinací znaků
	\end{itemize}
\end{enumerate}

Vylepšení:
\begin{itemize}
\item počítat se vznikem chyb (časté chyby, blízkost kláves, ..)
\item pravopisné chyby, ne jen překlep (mně x mě, shoda podmětu s přísudkem)
\item heuristicky (či strojové učení) na neznámá slova
\item vzít v potaz i kontext slova
\item na základě confidence chyby jeden z možných návrhů:
\begin{enumerate}
	\item nic nenabízet
	\item chcete A, nebo B
	\item potvrďte B
	\item auto oprava na B
\end{enumerate}
\end{itemize}

\subsection*{ASIMUT}
Automatická Selekce Informací Metodou Úplného Textu

\subsubsection*{Vyhledávací modul} 
Jazyk pro vyhledávání, např. \textit{vzdálenost!, odstup! -3- rodinný! -1- domek!}. Vlastně docela triviální.

\subsubsection*{Jazykový modul}
\begin{itemize}
\item pracuje s předpokladem, že slova se stejnou koncovkou mají stejné skloňování
\item složen z retrográdního slovníku, seznamu vzorů a výjimek
\item algoritmus 
\begin{itemize}
\item odzadu porovnávej vstupní slovo až do doby, než je jasné, jak se dané slovo skloňuje (popř. výjimka)
\item speciální kódování české diakritiky (pomocí čísel)
\item lze sepsat do pravidel v binárním stromu
\end{itemize}
\item výhoda: dobrej nápad
\item nevýhoda:
\begin{itemize}
\item počet výjimek může být příliš velký
\item není vždy jasné určit vzor na základě jen koncovky a málo pravidel
\item příliš hrubá klasifikace $\rightarrow$ příliš mnoho možných koncovek $\rightarrow$ přegenerování
\item verze pro slovesa funguje ještě hůř (časování)
\end{itemize}
\item \imwc{negativní slovník}{obsahuje slova, která nejsou důležitá pro dotazování (spojky, citoslovce apod.) a jsou v první fázi odstraněna z textu}
\item \imwc{konkordance}{všem důležitým slovním tvarům se přiřadila adresa a frekvence výskytu a pak se hledalo jen na konkordanci; slova v negativním slovníku měla adresu, ale nulovou frekvenci pro určení vzdálenosti mezi významovými slovy}
\end{itemize}

\subsection*{MOZAIKA}
\begin{itemize}
\item MOSAIC (Morphemic Oriented System of Automatic Indexing and Condensation)
\item indexace obvykle řešena pomocí slovníku klíčových slov + v dokumentu spočtena četnost
\item MOZAIKA řeší relevanci + více pojmů pro jeden denotát
\item založeno na pozorování části gramatiky: \textit{-or, -er} je konatel děje, \textit{-ity, -ness} vlastnost apod. (en)
\item je třeba pro každou tématickou oblast (např. elektrické inženýrství) vytvořit slovník takových přípon
\end{itemize}

Algoritmus: \begin{itemize}
\item vstup je text i s formátováním
\item projde to lemmatizací a morfologickou analýzou
\item pokud je slovo irelevantní k danému tématu, tak se vyhodí (negativní slovník)
\item kondenzace jmenných skupin pomocí jednoduché gramatiky (operační zesilovač TESLA KC 415 $\rightarrow$ zesilovač (s vyšší váhou)
\item váhy na základě pozice výskytů v textu (uprostřed nejméně důležité, na konci a začátku textu nejvíce)
\item normalizace dle délky dokumentů
\item výstup je seznam deseti nejvýznamnějších termínů v dokumentu $\rightarrow$ další zpracování pro vnější účely
\end{itemize}

Výhody: \begin{itemize}
\item není nutné vytvářet masivní slovník klíčových slov
\item kondenzace jmenných skupin je chytré
\end{itemize}

Nevýhody: \begin{itemize}
\item pracné vytváření tématických slovníků a pravidel
\item zájmena by měla zvyšovat četnost, ale to MOZAIKA nedetekuje
\end{itemize}

\pagebreak
\section*{Syntax}
\subsubsection*{Závislostní strom}
\begin{itemize}
\item rozumný pro jazyky s volným slovosledem (zejména slovanské)
\item přehledný, zachycuje zřejmé syntaktické vztahu mezi členy
\item neříká však nic o tom, jak sestrojit
\item ne všechny vztahy jsou řídící a podřízený (např. Petr a Pavel)
\item lze zploštit
\end{itemize}


\subsubsection*{Složkový strom}
\begin{itemize}
\item vhodný pro jazyky s pevným slovosledem
\item méně přehledný, obsahuje uzly, které nejsou větné členy, předpokládá bezkontext
\item lze automaticky zpracovat lépe (lze třeba uzávorkovat)
\item lze zploštit
\end{itemize}

\subsubsection*{Neprojektivní konstrukce}
\begin{itemize}
\item obvykle uspořádáváme vrcholy tak, jak jsou ve větě
\item strom nad danou větou je neprojektivní, pokud obsahuje neprojektivní závislost
\item \imwc{neprojektivní závislost}{závislost mezi dvěma slovy oddělenými ve větě třetím slovem, které (ani nepřímo) nezávisí na žádném z nich}
\item při splácnutí nelze nakreslit šipky bez křížení, nelze uzávorkovat
\item existují v mnoha jazycích
\end{itemize}

\subsection*{Transformační gramatika}
Historie:
\begin{itemize}
\item Deskriptivismus: popis a klasifikace faktů, ale nikoliv vysvětlení (povrchové)
\item Analytická syntax
\item Logický přístup: surface \& deep structure (různé výrazy mohou reprezentovat stejný význam, ale zároveň význam může mít více reprezentujících výrazů)
\end{itemize}

\subsection*{Noam Chomsky}
3 komponenty: \begin{itemize}
\item \imwc{Báze}{bezkontextová pravidla generující složkové stromy (phrase markers)}
\item \imwc{Transformační komponent}{pravidla operující na celých úrovních, vytváří povrchovou strukturu věty}
\item \imwc{Fonologický komponent}{pravidla generující fonetické interpretace}
\end{itemize}

\subsubsection*{Tree Adjoining Grammars}
\begin{itemize}
\item podobná myšlenka přepisování, ale přepisují se celé stromy, nikoliv jen řetězce
\item při substituci se musí vkládaný kořen a nahrazovaný list shodovat
\item algoritmus končí až když už nelze nic substituovat (nebo není třeba)
\item síla bezkontextových gramatik, ale lze udělat i silnější (s modifikacemi)
\end{itemize}

\subsubsection*{Lexial Functional Grammar}
\begin{itemize}
\item c-structures: spojují slova do frází
\item f-structures: reprezentace funkčních vztahů ve větě (např. shoda), matice atribut-hodnota
\item každá c-struktura spojena s pouze jednou f-strukturou, ale ne naopak
\end{itemize}

\subsubsection*{Kategoriální gramatiky}
\begin{itemize}
\item každý slovní tvar má přiřazen kategorii (např. [S\\NP]/NP)
\item malé množství pravidel typu: [X/Y => X]
\end{itemize}

\subsubsection*{Unifikační gramatiky}
\begin{itemize}
\item každý objekt má seznam vlastností/atributů (v neuspořádané množině)
\item unifikace je pak spojování tohoto seznamu; je dovolena pouze pokud nejsou konfliktní atributy (jinak sporná sestava)
\item spojují se však vlastnosti, které spolu nesouvisí, takže výrazy mají u sebe nerelevantní údaje
\text{} \tab řešení: typované sestavy rysů (typ určuje možné vlastnosti)
\item sestavy rysů pak obsahují kombinaci rysů, jenž popisují konkrétní jev (např. shoda)
\end{itemize}

\subsubsection*{HPSG}
Hot Potato Soup with Garlic \\
Generování řetězců pomocí pozice v hiearchii

\subsubsection*{Augmented Transition Networks}
3 typy hran, přechod mezi sítěmi

\subsection*{Q-systémy}
\begin{itemize}
\item transformace grafů (vět)
\item stromy jsou linearizovány (s neprojektivními konstrukcemi se dá nějak vypořádat)
\item přesně definované typy objektů: atomy, stromy a seznamy stromů
\item jednopísmenné proměnné, logické operátory
\item spojování vrcholů, gramatická pravidla
\end{itemize}

\subsection*{Funkční generativní popis, valence}
\begin{itemize}
\item ještě větší štěpení: 5 rovin (fonetická, fonologická, morfématická, povrchová, tektogramatická)
\item tektogramatická rovina nejvyšší, závislosti
\item valence (povolené kombinace různých členů - aktantů), Vallex \begin{itemize}
	\item vytváří formy (Lojban)
	\item TG rovina má členy: konator/aktor, patient, adresát, origo, efekt (každý pouze jednou, ale lze triviálně koordinovat)
	\item každý aktant může být ještě obligatorní/fakultativní (obligatorní nesmí chybět)
	\item aktant je obligatorní, pokud nemůžeme odpovědět na otázku pomocí \textit{nevím}
	\item \textit{Moji přátelé přijeli}. Kam? \textbf{Nevím}, Odkud? Nevím. \textit{Kam} tedy má obligatorního aktanta
\end{itemize}
\end{itemize}

\subsection*{Kontrola gramatiky}
\begin{itemize}
	\item dobrá heuristika je řešit pouze nejčastější chyby
	\item nechceme žádné false positive (precision blízko 1, na recall kašleme)
	\item jazyky s volným slovosledem jsou problematické
\end{itemize}

\subsubsection*{RFODG}
Robust Free-Order Dependency Grammar je použití ručně psaných sjednocovacích gramatik. Různé fáze výpočtu (pozitivní projektivní, negativní projektivní/pozitivní neprojektivní, negativní neprojektivní). Různé gramatiky mohou popisovat i chybné věty.

\subsubsection*{LanGR}
Pravidla opět psaná ručně. Nejsou aplikována systematicky (neuspořádaně), ale cyklicky. 

\section*{Automatický překlad přirozených jazyků}
Kromě samotného slovníku slov je třeba znát tvarosloví (morfologie), pravopis (syntax), ustálená spojení (idiomy). Slovníky však nejsou 1:1, neboť reflektují kulturu. Závisí taky na kontextu z předchozích vět (podobný problém lemmatizace).

Obvykle se posunujeme v analýze směrem k interlingvě a v nějakém bodě provedeme transfer a vygenerujeme cílový text.

\subsubsection*{Historie}
Už od války - první vyvrcholení v USA utnuto ALPACem, pokračování v Evropě. Žádně extra dobré výsledky. Řešením bylo omezit doménu. Systém TAUM - METEO (vytvořili k tomu Q-systémy) první úspěšný. Pak pokusy s SYSTRAN a EUROTRA. VERBMOBIL chtěl kromě překladu dělat rozpoznávání a syntézu řeči (rok 2000). 




\section*{Konec}
Nějak tak jsem si uvědomil, že pěkněji zpracované poznámky jsou na \href{http://wiki.matfyz.cz/wiki/PFL012-pozn\%C3\%A1mky}{mff wiki}. Obsah přednášek se moc nemění během let, tudíž je asi rozumnější řídit se tím.
\end{document}