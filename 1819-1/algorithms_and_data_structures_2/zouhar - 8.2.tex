\documentclass[a4paper]{article}
\usepackage[english]{babel}
\usepackage[utf8x]{inputenc}
\usepackage[T1]{fontenc}
\usepackage{listings}
\usepackage[a4paper,margin=2cm]{geometry}
\usepackage{amsmath}
\usepackage{graphicx}
\usepackage[colorinlistoftodos]{todonotes}
\usepackage[colorlinks=true, allcolors=blue]{hyperref}
\usepackage{wasysym} % smileys
\usepackage{fancyhdr}
\usepackage{tikz}
\usetikzlibrary{arrows}
\setlength\parindent{0pt} % indent


% my commands:
\newcommand{\n}{\newline}
\newcommand{\tab}{\hspace{1cm}}

\begin{document}

\renewcommand{\headrulewidth}{0pt} % removes horizontal bars from headers and footers
\thispagestyle{fancy} % beware the difference between \thispagestyle and \pagestyle
\lhead{8.2}
\rhead{Vilém Zouhar}

\section*{BMerge}
\subsection*{a)}
Pro $n = 1$ (nenastane) vracíme $x_0$, pro $n \le 2$ pak $min(x_0, x_1), max(x_0, x_1)$ a je setřízeno. Pokud začneme od listů (takhle bude stejně konstruovaná síť), pak z indukce stačí rozhodnout už jen obecný případ. Jestliže máme v $(a_0, \ldots, a_{n-1})$ a $(b_0, \ldots, b_{n-1})$ setřízené, pak máme setřízené $(x_0, \ldots, x_{n-1})$ jednotlivě na sudých a lichých místech. Pak spouštíme porovnání $C_i$ na $C(x_{2i}, x_{2i+1}) \forall i \in \frac{n}{2})$ a podle velikosti je prohodíme. Po tomto porovnání budou tyto dvě čísla seřazená, ale nutně musí být i seřazená s dvojicí vedle, neboť $x_{2i+1} \le x_{2i+3} \wedge x_{2i} \le x_{2i+2}$. Můžeme to rozebrat po příkladech:
\begin{itemize}
\item $C_i$, $C_{i+2}$ prohodí, pak nemůže být $x_{2i} \ge x_{2i+3}$, jinak by $x_{2i+1} \ge x_{2i+3}$ \\
\item $C_i$, $C_{i+2}$ neprohodí, pak nemůže být $x_{2i+1} \ge x_{2i+2}$ opět z tranzitivity \\
\item $C_i$ prohodí, $C_{i+2}$ neprohodí, nebo naopak, pak nemůže být $x_{2i} \ge x_{2i+2}$, resp  $x_{2i+1} \ge x_{2i+3}$ přímo z definice setřízených posloupností \\
\end{itemize}

\subsection*{b)}
Pokud máme vstup délky $n$, pak uděláme dvě subsítě pro liché a sudé prvky a přidáme mezi každé obdvojice komparátor (prvky této obdvojice však nemusí být v síti vedle sebe; subsítě jsou proloženy přes sebe). Triviální komparátor použijeme pro $n=2$.

\end{document}