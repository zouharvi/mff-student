\documentclass[a4paper]{article}
\usepackage[english]{babel}
\usepackage[utf8x]{inputenc}
\usepackage[T1]{fontenc}
\usepackage{listings}
\usepackage[a4paper,margin=2cm]{geometry}
\usepackage{amsmath}
\usepackage{graphicx}
\usepackage[colorinlistoftodos]{todonotes}
\usepackage[colorlinks=true, allcolors=blue]{hyperref}
\usepackage{wasysym} % smileys
\usepackage{fancyhdr}
\usepackage{tikz}
\usetikzlibrary{arrows}
\setlength\parindent{0pt} % indent


% my commands:
\newcommand{\n}{\newline}
\newcommand{\tab}{\hspace{1cm}}

\begin{document}

\renewcommand{\headrulewidth}{0pt} % removes horizontal bars from headers and footers
\thispagestyle{fancy} % beware the difference between \thispagestyle and \pagestyle
\lhead{10.1}
\rhead{Vilém Zouhar}

\section*{NzMn}
Nejprve zjistíme velikost největší nezávislé množiny, formálně jako $K = argmax_{k \in n..0} \{ NzMn(G,k) \}$ (očekáváme, že argmax to prochází uspořádaně a zastaví se na prvním maximu. Iterativně by to šlo zapsat jako:

\begin{lstlisting}
for K in n..0:
	if NzMn(G, K) == 1:
		break
\end{lstlisting}

Následně budeme procházet všechny vrcholy a u každého rozhodneme, zdali existuje nezávislá množina velikosti $K$ s ním, nebo bez něj. V kódu by šlo zapsat:

\begin{lstlisting}
for i in 1..n:
	if NzMn(G \ i, K) == 1:
		G = G \ v_i		
out (G)
\end{lstlisting}

Pokud tedy u nějakého vrcholu zjistíme, že neovlivňuje existenci nezávislé množiny velikosti $K$, pak jej můžeme z grafu odebrat. Nakonec v grafu zůstanou jen vrcholy největší nezávislé množiny.

Obojí nás jistě stálo $O(n \cdot T)$, tudíž jsme polynomiální vzhledem k $T$.
\end{document}