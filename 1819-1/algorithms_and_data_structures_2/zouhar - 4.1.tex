\documentclass[a4paper]{article}
\usepackage[english]{babel}
\usepackage[utf8x]{inputenc}
\usepackage[T1]{fontenc}
\usepackage{listings}
\usepackage[a4paper,margin=2cm]{geometry}
\usepackage{amsmath}
\usepackage{graphicx}
\usepackage[colorinlistoftodos]{todonotes}
\usepackage[colorlinks=true, allcolors=blue]{hyperref}
\usepackage{wasysym} % smileys
\usepackage{fancyhdr}
\usepackage{tikz}
\usetikzlibrary{arrows}
\setlength\parindent{0pt} % indent

% my commands:
\newcommand{\n}{\newline}
\newcommand{\tab}{\hspace{1cm}}

\begin{document}

\renewcommand{\headrulewidth}{0pt} % removes horizontal bars from headers and footers
\thispagestyle{fancy} % beware the difference between \thispagestyle and \pagestyle
\lhead{4.1}
\rhead{Vilém Zouhar}

\section*{Popis}
Budeme vycházet z myšlenky základní verze, ve které jsme z řádků udělali jednu partitu a ze sloupců druhou. Nyní to však nebudou celé řádky a sloupce, ale řádkové a sloupcové segmenty. Formálně tedy každý řádek rozdělíme na segmenty (maximální úseky bez děr) a ty vložíme do množiny $A$. Obdobně se sloupci, jejichž segmenty vložíme do množiny $B$. Hrany mezi $a \in A$ a $b \in B$ existují tehdy, pokud se segmenty kříží. Na výsledném bipartitním grafu stačí nalézt maximální párování (hrana značí, že se na křížícím místě položí věž).

\section*{Důkaz} 
Nalezené rozestavení je validní, neboť pokud by nebylo, tak jsme umístili věže tak, že se ohrožují, tedy nachází se ve stejném segmentu. To by však znamenalo, že z vrcholu (reprezentující daný segment) vedou alespoň dvě hrany, což je spor s definicí párování. Naopak každé validní rozestavení má ekvivalentní párování v naší reprezentaci. Po obměně: jestliže není něco v naší reprezentaci párování, tak to není validní rozestavení. Opět platí argument s více hranami z jednoho vrcholu.


\section*{Algoritmus a složitost}
Segmentů může vzniknout až $O(n^2)$, hran též ($n$ je šířka/výška desky). Pro hledání maximálního párování použijeme nějaký algoritmus největšího toku (zapojíme jednu partitu do zdroje, druhou do stoku a všemu dáme kapacitu 1). Nejefektivnější umí řešit až v $O(VE)$, tedy celkem $O(n^4)$.
\\

PS: Až po dopsání jsem si všiml, že se jedná o šachovnici ($8 \times 8$). Tedy je vše konstanta a náš program seběhne v $O(1)$.

\end{document}