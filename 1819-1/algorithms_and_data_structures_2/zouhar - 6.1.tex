\documentclass[a4paper]{article}
\usepackage[english]{babel}
\usepackage[utf8x]{inputenc}
\usepackage[T1]{fontenc}
\usepackage{listings}
\usepackage[a4paper,margin=2cm]{geometry}
\usepackage{amsmath}
\usepackage{graphicx}
\usepackage[colorinlistoftodos]{todonotes}
\usepackage[colorlinks=true, allcolors=blue]{hyperref}
\usepackage{wasysym} % smileys
\usepackage{fancyhdr}
\usepackage{tikz}
\usetikzlibrary{arrows}
\setlength\parindent{0pt} % indent


% my commands:
\newcommand{\n}{\newline}
\newcommand{\tab}{\hspace{1cm}}

\begin{document}

\renewcommand{\headrulewidth}{0pt} % removes horizontal bars from headers and footers
\thispagestyle{fancy} % beware the difference between \thispagestyle and \pagestyle
\lhead{6.1}
\rhead{Vilém Zouhar}

\section*{Goldberg}
Toto nechci odevzdávat jako úkol, jen by mě zajímalo, zdali to lze řešit touto strukturou. Uhádat složitosti se mi totiž nepodařilo. Klíčové je uchovávání aktivních vrcholů. Ty si budeme uchovávat v poli indexovaném výškami. V $i$-tém políčku $b$ tedy bude první prvek spojového seznamu vrcholů s výškou $i$. Graf jinak reprezentujeme seznamem sousedů. Používáme variantu, kdy vybíráme nejvyšší aktivní vrchol a zvyšujeme vždy o jedna více, než minimum vhodného souseda.


\end{document}