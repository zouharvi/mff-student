\documentclass[a4paper]{article}
\usepackage[english]{babel}
\usepackage[utf8x]{inputenc}
\usepackage[T1]{fontenc}
\usepackage{listings}
\usepackage[a4paper,margin=2cm]{geometry}
\usepackage{amsmath}
\usepackage{graphicx}
\usepackage[colorinlistoftodos]{todonotes}
\usepackage[colorlinks=true, allcolors=blue]{hyperref}
\usepackage{wasysym} % smileys
\usepackage{fancyhdr}
\usepackage{tikz}
\usetikzlibrary{arrows}
\setlength\parindent{0pt} % indent

% my commands:
\newcommand{\n}{\newline}
\newcommand{\tab}{\hspace{1cm}}

\begin{document}

\renewcommand{\headrulewidth}{0pt} % removes horizontal bars from headers and footers
\thispagestyle{fancy} % beware the difference between \thispagestyle and \pagestyle
\lhead{3.2}
\rhead{Vilém Zouhar}

\section*{Popis}
Pakliže $f_m (e_1) \le c'(e_1)$, tak $f'_m = f_m$. Jinak pro $e_1 = (a,b)$ vezmeme $b$ a pomocí $BFS$ najdeme cestu $P_s$ do stoku, obdobně pro $a$ nalezneme cestu $P_z$ do zdroje. Na těchto cestách snížíme tok o 1, čímž dostaneme validní tok (nemusí být maximální). Aby byl maximální, tak spustíme na reziduálním grafu (lze in-place původního) ze zdroje do stoku BFS, které pokud nalezne zlepšující cestu, tak můžeme vylepšit (o 1). Tím jsme skončili, neboť $|f_m| = |f'_m|$ a maximální tok nového grafu nemůže být větší, neboť je slabší. Pokud nenajdeme žádnou zlepšující cestu, tak jsme taky skončili, neboť platí lemma o souvislosti existence zlepšující cesty a maximalitou toku.


\section*{$P_z$, $P_s$}
Na výpočet těchto cest nepoužíváme jen původní graf, ale využíváme hrany, po kterých teče alespoň komodita velikosti $1$. Existenci těchto cest pak máme zaručenou z Kirchhofova zákona.

\section*{Složitost}
Na výpočet $P_z$, $P_s$ potřebujeme $O(m+n)$. To stejné pro výpočet zlepšující cesty, tedy zároveň je to i složitost řešení.

\end{document}