\documentclass[a4paper]{article}
\usepackage[english]{babel}
\usepackage[utf8x]{inputenc}
\usepackage[T1]{fontenc}
\usepackage{listings}
\usepackage[a4paper,margin=2cm]{geometry}
\usepackage{amsmath}
\usepackage{graphicx}
\usepackage[colorinlistoftodos]{todonotes}
\usepackage[colorlinks=true, allcolors=blue]{hyperref}
\usepackage{wasysym} % smileys
\usepackage{fancyhdr}
\usepackage{tikz}
\usetikzlibrary{arrows}
\setlength\parindent{0pt} % indent

% my commands:
\newcommand{\n}{\newline}
\newcommand{\tab}{\hspace{1cm}}

\begin{document}

\renewcommand{\headrulewidth}{0pt} % removes horizontal bars from headers and footers
\thispagestyle{fancy} % beware the difference between \thispagestyle and \pagestyle
\lhead{1.1}
\rhead{Vilém Zouhar}

\section*{Popis}
Vstup si pojmenujeme $T = AB$ s tím, že chceme vytvořit výstup $T = BA$. Velikosti si označíme: $|A| = a, |B| = b$. Algoritmus funguje jednoduše tak, že se snaží opravit místo $T[a]$. Definujeme si funkci $where(i)$, která vrací hodnotu, kam patří znak z $i-$té pozice v původním řetězci. Ta vypadá následovně. Slovně všechny znaky v $A$ posune o $b$ doprava a opačně pro znaky v $B$.
\begin{lstlisting}
def where(i):
	if i < a:
		return b + i
	else:
		return i - a
\end{lstlisting}

Nyní se budeme soustředit na prvek $T[a]$, ze kterého budeme napravovat zbytek řetězce. Začneme $swap(a, where(a))$, což je samozřejmě to stejné, jako $swap(a, 0)$. Vyhodnocené číslo $0$ si uložíme do proměnné $t$. Na indexu $0$ je už správný znak, takže se zde už swapovat s ničím nebude. Následně jen $(a+b)-$krát provedeme $\{ swap(a, where(t)); t = where(t); \}$

\section*{Pseudokód}
\begin{lstlisting}
def where(i):
	if i < a:
		return b + i
	else:
		return i - a

t = a
for i in 0..a+b:
	swap(a, where(t))
	t = where(t)
\end{lstlisting}
Funkce $swap$ je definovaná očekávaným způsobem. Pracuje na konstantní paměti v konstantním čase.

\section*{Složitost}
Děláme konstantní práci na $\textit{každém znak}$. Výsledkem je $O(a+b)$. Používáme jen konstantní paměť navíc.

\section*{Správnost}
Pro $a=b$ tento postup selže, neboť hned na začátku si na $T[a]$ nastavíme správnou hodnotu, takže by se tento případ musel ošetřit zvlášť. Místo důkazu správnosti zbylých případů mě napadlo jiné, snad rozumnější, řešení (druhá strana).

\pagebreak

\section*{Popis}
Tento algoritmus pracuje s překlápěním. Nejprve si překlopíme celé pole a pak jednotlivé části.

\section*{Pseudokód}
\begin{lstlisting}
def flip(i, j):
	for z in i..j/2:
		swap(z, j-z)

flip(0, a+b)
flip(0, b)
flip(b+1, a+b)
\end{lstlisting}

\section*{Složitost}
Na každém znaku uděláme dvě konstantní operace. Paměť navíc, kromě pro swapy, nepotřebujeme. Celkem $O(a+b)$ časově a $O(1)$ paměťově.

\section*{Správnost}
Můžeme se podívat, co se stane s prvkem původně: $T[x]$. Pro případ $x < a$ nejprve dostaneme $T[a+b-x]$, ale pak $T[b + a+b - a - b + x]\hspace{0.2cm}..\hspace{0.2cm}T[b+x]$, což jsme chtěli. Symetricky pro $x \ge a$.

\end{document}