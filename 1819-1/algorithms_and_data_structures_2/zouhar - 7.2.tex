\documentclass[a4paper]{article}
\usepackage[english]{babel}
\usepackage[utf8x]{inputenc}
\usepackage[T1]{fontenc}
\usepackage{listings}
\usepackage[a4paper,margin=2cm]{geometry}
\usepackage{amsmath}
\usepackage{graphicx}
\usepackage[colorinlistoftodos]{todonotes}
\usepackage[colorlinks=true, allcolors=blue]{hyperref}
\usepackage{wasysym} % smileys
\usepackage{fancyhdr}
\usepackage{tikz}
\usetikzlibrary{arrows}
\setlength\parindent{0pt} % indent


% my commands:
\newcommand{\n}{\newline}
\newcommand{\tab}{\hspace{1cm}}

\begin{document}

\renewcommand{\headrulewidth}{0pt} % removes horizontal bars from headers and footers
\thispagestyle{fancy} % beware the difference between \thispagestyle and \pagestyle
\lhead{7.2}
\rhead{Vilém Zouhar}

\section*{XOR}
Za toto jsem dostal zápočet na IPS I., tak nevím, jak moc se to sem hodí. Zprávu $N$ zašifrujeme do čísla $Z$ o základu $2$ (triviálně umíme skoro vždy). Pro každou $k$-tici generálů $G_i$ uděláme následovné:
\begin{enumerate}
	\item každý generál $l$ dostane náhodné binární číslo $g_i^l$ délky $|Z|$
	\item číslo $Z_i = Z \oplus g_i^1 \oplus g_i^2 \oplus \ldots \oplus g_i^k$ zveřejníme
	\item zveřejníme všech seznam generálů ve všech ${n \choose k}$ skupinách
\end{enumerate}
\text{}\\
Pak zřejmě:
\begin{enumerate}
	\item pokud se sejde nějaká $k$-tice $G_i$, pak mají k dispozici čísla $Z_i, g_i^1, g_i^2, \ldots, g_i^k$
	\item zprávu $Z$ dekonstruují jako $Z = Z_i \oplus g_i^1 \oplus g_i^2 \oplus \ldots \oplus g_i^k$
	\item stačí již dešifrovat zprávu $Z$ do původní $N$ 
\end{enumerate}

Kritická část je šifrování pomocí $XOR$, ale:
\begin{align*}
	& Z_i = Z \oplus g_i^1 \oplus g_i^2 \oplus \ldots \oplus g_i^k = Z \oplus Q_k \\
	& Z_i \oplus Q_k = Z \oplus Q_k \oplus Q_k = Z \oplus 0 = Z
\end{align*}

Nevýhodou je, že musíme vygenerovat ${n \choose k} \cdot k \cdot |Z|$ bitů a ty rozdistribuovat. Pokud se to však podaří, pak neprozradíme ani jeden bit informace v procesu (\textit{not a bit}), neboť $XOR$ náhodným číslem je opět náhodné číslo..

\section*{Algebraický přístup}
Předchozí řešení je dost technické. Úloha lze snad řešit i víc matematicky. Předpokládáme, že zprávu $N$ zašifrujeme do $k$-dimenzionálního prostoru (např. rozdělíme na $n$ částí). Pak vygenerujeme $n$ různých nadrovin, které procházejí tímto bodem a jejich parametry rozdáme generálům. Jestliže se jich sejde alespoň $k$, tak jejich nadroviny jednoznačně určí původní $N$ (méně určí nadrovinu o dimenzi nižší, což však nestačí).

Nevýhodou je, že předpokládáme zašifrovatelnost do bodu v nějakém prostoru. To je mnohem silnější předpoklad než u $XOR$ řešení. Pokud však můžeme o $N$ něco předpokládat, tak $k-1$ generálů má nadrovinu o jedna nižší dimenzi, kde se zpráva nachází, což může být problém.

Druhé řešení se mi líbí v tom, že každý generál dostane pouze jednu sadu parametrů, nikoliv pro každou možnou skupinu.
\end{document}