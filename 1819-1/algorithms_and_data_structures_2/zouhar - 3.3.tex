\documentclass[a4paper]{article}
\usepackage[english]{babel}
\usepackage[utf8x]{inputenc}
\usepackage[T1]{fontenc}
\usepackage{listings}
\usepackage[a4paper,margin=2cm]{geometry}
\usepackage{amsmath}
\usepackage{graphicx}
\usepackage[colorinlistoftodos]{todonotes}
\usepackage[colorlinks=true, allcolors=blue]{hyperref}
\usepackage{wasysym} % smileys
\usepackage{fancyhdr}
\usepackage{tikz}
\usetikzlibrary{arrows}
\setlength\parindent{0pt} % indent

% my commands:
\newcommand{\n}{\newline}
\newcommand{\tab}{\hspace{1cm}}

\begin{document}

\renewcommand{\headrulewidth}{0pt} % removes horizontal bars from headers and footers
\thispagestyle{fancy} % beware the difference between \thispagestyle and \pagestyle
\lhead{3.3}
\rhead{Vilém Zouhar}

\section*{Popis}
Nejprve nalezneme minimální $A-B$ řez. Z něj pak odstraníme $k$ hran s největším průtokem (přiřazený tokem $f_m$).

\section*{Důkaz}
Nemám to rozmyšlené. Je to spíše myšlenka u které by mě zajímalo, zdali jde správným směrem.


\section*{Řez}
Řez nalezneme v reziduálním grafu pomocí výpočtu maximálního toku. Ze zdroje prohledáváme třeba BFS, čímž dostaneme množinu $A$ (do stoku se dostat nemůžeme, jinak bychom právě nalezli zlepšující cestu) a zbytek vrcholů je $B$.

\section*{Složitost}
Výpočet řezu trvá $O(n(n+m) + n+m)$, pro získání $k$ $A-B$ hran je potřebujeme seřadit, což je $O(m\cdot log(m)) = O(m\cdot log(n))$. Celkově tedy stále $O(n(n+m))$.

\end{document}