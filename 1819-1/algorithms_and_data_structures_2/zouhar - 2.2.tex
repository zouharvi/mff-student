\documentclass[a4paper]{article}
\usepackage[english]{babel}
\usepackage[utf8x]{inputenc}
\usepackage[T1]{fontenc}
\usepackage{listings}
\usepackage[a4paper,margin=2cm]{geometry}
\usepackage{amsmath}
\usepackage{graphicx}
\usepackage[colorinlistoftodos]{todonotes}
\usepackage[colorlinks=true, allcolors=blue]{hyperref}
\usepackage{wasysym} % smileys
\usepackage{fancyhdr}
\usepackage{tikz}
\usetikzlibrary{arrows}
\setlength\parindent{0pt} % indent

% my commands:
\newcommand{\n}{\newline}
\newcommand{\tab}{\hspace{1cm}}

\begin{document}

\renewcommand{\headrulewidth}{0pt} % removes horizontal bars from headers and footers
\thispagestyle{fancy} % beware the difference between \thispagestyle and \pagestyle
\lhead{2.2}
\rhead{Vilém Zouhar}

\section*{Popis}
\begin{enumerate}
\item 
Ze vzorků $S_i$ si vytvoříme "automat" ve smyslu AC. Vybudujeme zároveň i zpětnou funkci+ ($zh+$), která tranzitivně bude obsahovat zpětné hrany, ale vrcholy, které nejsou koncem žádného vzorku, přeskočí. U každého stavu si též budeme pamatovat, jaké zpětné hrany+ na něj míří. Ve stavech, ze kterých žádná $zh+$ hrana nevede, přidáme takovou hranu do stavu $0$. Pokud nalezneme nějaký výskyt, tak si v aktuálním stavu inkrementujeme akumulátor a takto projdeme celý text. Četnost stavu $0$ ponecháme na $0$.
\item
Na konci zavoláme funkci $cclt$ na kořen, pak ve stromě nalezneme vzorky a vyzvedneme si u nich jejich četnost. Funkce $cclt$ má rekurzivní charakter a je definována následovně:
\begin{lstlisting}
def cclt(s):
	for state k in s.rev_zhp:
		freq[s] += cclt(k);
	return freq[state]
\end{lstlisting}
\end{enumerate}

\section*{Korektnost}
Každý běh $cclt$ někdy skončí, neboť máme konečný počet vzorků (tedy i stavů, kde může vzorek konči) a k cyklení dojít nemůže, neboť stavy, mezi kterými vede $zh+$ hrana nemůžou mít stejnou velikost. Zároveň taky nevynecháme žádné výskyty, neboť pokud ze stavu $s$ vedou opačně $zh+$ hrany k množině stavů $A$, pak neexistují jiné vzorky, než v $A$, které by mohly přispět k četnosti $s$.

Předpokládáme, že $zh+$ je intrazitivní, resp. tvoří strom. Jinak by vznikl problém s pořadím zpracování. To je však zřejmé, neboť pokud by tuhle vlastnost neměla, tak z nějakého stavu by musely vést dvě $zh+$, což je ve sporu s konstrukcí.

\section*{Složitost}
1. fáze trvá $O(T+\sum S_i)$, neboť se jedná jen o lehce upraveného AC a konstrukci $zh+$ jsme řešili na cvičení. 2. fáze trvá jen dokonce $O(|S|)$, neboť děláme konstantní práci na každém vrcholu, kde končí vzorek. Celkem $O(T+\sum S_i)$, neboť $\sum S_i \ge \sum_{S_i} 1 = |S|$.

\end{document}