\documentclass[a4paper]{article}
\usepackage[english]{babel}
\usepackage[utf8x]{inputenc}
\usepackage[T1]{fontenc}
\usepackage{listings}
\usepackage[a4paper,margin=2cm]{geometry}
\usepackage{amsmath}
\usepackage{graphicx}
\usepackage[colorinlistoftodos]{todonotes}
\usepackage[colorlinks=true, allcolors=blue]{hyperref}
\usepackage{wasysym} % smileys
\usepackage{fancyhdr}
\usepackage{tikz}
\usetikzlibrary{arrows}
\setlength\parindent{0pt} % indent

% my commands:
\newcommand{\n}{\newline}
\newcommand{\tab}{\hspace{1cm}}

\begin{document}

\renewcommand{\headrulewidth}{0pt} % removes horizontal bars from headers and footers
\thispagestyle{fancy} % beware the difference between \thispagestyle and \pagestyle
\lhead{1.3}
\rhead{Vilém Zouhar}

\section*{Popis}
Ze vstupního slova $S$ si vytvoříme vyhledávací automat, jako v Aho-Corasick. Samotný automat nám je k ničemu, ale při jeho budování se zachovává invariant o zpětné funkci, která například stavu $A$ přiřazuje $B$. Slovo $S_B$, tvořené cestou z kořene do stavu $B$, je prefix celého slova $S$, ale zároveň se jedná o sufix $S_A$. Můžeme tedy stavit vyhledávací automat pro $S$ a podívat se, kam vede zpětná funkce z konečného stavu. Zcela jistě vede do nějakého stavu $I$, který je prefixem celého slova $S$ a zároveň jeho sufixem, neboť jsme sem skočili z koncového stavu.

\section*{Korektnost}
Vlastnosti návratové funkce jsme probírali na přednášce.

\section*{Složitost}
Stavba automatu trvá $O(|S|)$, avšak i vyhledávání a celkové zpracování. Taková je tedy složitost jak v čase, tak v paměti.

\end{document}