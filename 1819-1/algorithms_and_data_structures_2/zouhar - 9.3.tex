\documentclass[a4paper]{article}
\usepackage[english]{babel}
\usepackage[utf8x]{inputenc}
\usepackage[T1]{fontenc}
\usepackage{listings}
\usepackage[a4paper,margin=2cm]{geometry}
\usepackage{amsmath}
\usepackage{graphicx}
\usepackage[colorinlistoftodos]{todonotes}
\usepackage[colorlinks=true, allcolors=blue]{hyperref}
\usepackage{wasysym} % smileys
\usepackage{fancyhdr}
\usepackage{tikz}
\usetikzlibrary{arrows}
\setlength\parindent{0pt} % indent


% my commands:
\newcommand{\n}{\newline}
\newcommand{\tab}{\hspace{1cm}}

\begin{document}

\renewcommand{\headrulewidth}{0pt} % removes horizontal bars from headers and footers
\thispagestyle{fancy} % beware the difference between \thispagestyle and \pagestyle
\lhead{9.3}
\rhead{Vilém Zouhar}

\section*{Binární odčítačka}
Je to možná trochu cheatování, ale využil bych vlastnosti dvojkového doplňku. Tedy pokud máme na vstupu $X, Y$ a chceme $X - Y$, pak flipneme $Y$ (a přičteme 1) a použijeme binární sčítačku na $X, -Y$, výsledek pak $X + -Y = X - Y$. Sčítání použijeme tedy dvakrát a jeho hloubka je $O(log\ n)$. Počet hradel tedy bude stále $O(n^2)$.


\end{document}