% !TeX spellcheck = cs_CZ

\documentclass[a4paper]{article}
\usepackage[english]{babel}
\usepackage[utf8x]{inputenc}
\usepackage[T1]{fontenc}
\usepackage{listings}
\usepackage[a4paper,margin=2cm]{geometry}
\usepackage{amsmath}
\usepackage{graphicx}
\usepackage[colorlinks=true, allcolors=blue]{hyperref}
\usepackage{wasysym} % smileys
\usepackage{fancyhdr}
\setlength\parindent{0pt} % indent

% my commands:
\newcommand{\n}{\newline}
\newcommand{\tab}{\hspace{1cm}}

\begin{document}

\thispagestyle{fancy} % beware the difference between \thispagestyle and \pagestyle
\lhead{Dec 2018}
\rhead{Vilém Zouhar}

Neúplný seznam definic, vět a poznatků ke zkoušce z Výrokové a predikátové logiky

\newcommand{\bP}{$\mathbf{P}$}
\newcommand{\ph}{$\phi$}

\section{Úvod, tablo}

\subsection{Definice}
\begin{itemize}
\item četnost/arita relace a funkce, mohutnost
\item jazyk výrokové logiky, formule
\item množina všech výrokových formulí nad jazykem \bP
\item množina všech výrokových proměnných ve formuli \ph
\item ohodnocení výroku, vytvořující strom výroku
\item model, třída všech modelů; pravdivý, lživý, splnitelný, nezávislý, ekvivalentní model
\item CNF, DNF, literál, prázdná konjunkce/disjunkce, (elementární) konjunkce a klauzule
\item SAT
\item jednotková klauzule, Hornova klauzule, Hornův výrok
\item teorie, model teorie, třída modelů teorie
\item pravdivý, lživý, nezávislý, splnitelný, ekvivalentní v teorii
\item sporná, kompletní teorie, extenze (jednoduchá, konzervativní) teorie, ekvivalentní teorie
\item důsledek teorie
\item algebra výroků
\item korektnost a úplnost dokazovacích systémů
\item tablo, zamítnutí, důkaz, redukovaná položka, sporná, dokončená větev, dokončené, sporné tablo
\item atomické tablo, konečné tablo z [prázdné] teorie, systematické tablo
\item množina teorémů teorie (vs. důsledky)
\item sporná, kompletní, extenze (jednoduchá, konzervativní), ekvivalentní teorie
\end{itemize}

\subsection{Věty}
\begin{itemize}
\item Cantorova věta
\item Königovo lemma
\item 2-SAT polynomiálně + korektnost a úplnost
\item Horn-SAT
\item počty výroků
\item konečnost tablo důkazu
\item lemma o shodnosti větve s modelem
\item korektnost tablo důkazu
\item lemma o tvaru shodnosti větve s modelem 
\item úplnost tablo důkazu
\item věta o kompaktnosti
\end{itemize}

\pagebreak

\section{Rezoluční metoda, konec VL}

\subsection{Definice}
\begin{itemize}
\item literál, množinová reprezentace CNF, (částečné|totální) ohodnocení
\item rezoluční pravidlo, rezoluční důkaz
\item rezoluční strom, rezoluční uzávěr
\item redukce dosazením
\item lineární důkaz, (počáteční|centrální|boční) klauzule
\item fakt, pravidlo, cíl
\end{itemize}

\subsection{Věty}
\begin{itemize}
\item korektnost rezoluce
\item lemma: formule splnitelná právě když jedna z jejich redukcí \textit{l} je
\item důsledek: formule není splnitelná právě když každá větev obsahuje spor
\item úplnost rezoluce
\item korektnost LI rezoluce
\item úplnost LI rezoluce pro Hornovy formule
\end{itemize}

\pagebreak

\section{Úvod PL, tablo}

\subsection{Definice}
\begin{itemize}
\item jazyk, signatura jazyka, term, atomická formule, formule
\item (vázaný|volný) výskyt proměnné, (otevřené|uzavřené) formule
\item instance, substituovatelný term, varianta
\item struktura pro jazyk
\item hodnota (termu|(atomické|) formule), platnost formule při ohodnocení
\item platnost ve struktuře
\item teorie jazyka, model teorie, třída modelů teorie
\item (pravdivá|lživá|nezávislá) v teorii, důsledek teorie
\item (sporná|kompletní) teorie, (jednoduchá|konzervativní) extenze, (elementárně) ekvivalentní
\item (indukovaná|generovaná|) podstruktura, restrikce
\item (redukt|expanze) struktury do jazyka
\item definovatelné množiny
\item kongruence pro (funkci|relaci)
\item dokončené tablo, kanonický model (s rovností|)
\end{itemize}

\subsection{Věty}
\begin{itemize}
\item $T, \neg \phi $ nemá model $ \Leftrightarrow T \models \phi$
\item věta o konstantách
\item dokončenost systematického tabla PL
\item konečnost sporného tabla PL
\item význam axiomu rovnosti
\item lemma o shodnosti s kořenem
\item korektnost tablo metody pro PL
\item lemma o shodnosti kanonického modelu
\item úplnost tablo metody pro PL
\end{itemize}

\pagebreak

\section{Skolem, Herbrand}

\subsection{Definice}
\begin{itemize}
\item vytýkání kvantifikátorů
\item prenexní (normální|) tvar
\item Skolemova varianta
\item Herbrandův modely
\end{itemize}

\subsection{Věty}
\begin{itemize}
\item Löwenheim-Skolemova věta
\item kompaktnost v PL
\item věta o rozšiřování teorií
\item extenze o definici relace
\item extenze o definici funkce
\item extenze o definici (spousta lemmat a důsledků)
\item převody na prenexní tvar
\item vlastnosti Skolemovy varianty
\item Skolemova věta
\end{itemize}

\pagebreak


\section{Rezoluce PL}

\subsection{Definice}
\begin{itemize}
\item unifikace
\end{itemize}

\subsection{Věty}
\begin{itemize}
\item korektnost unifikace
\end{itemize}

\end{document}