% !TeX spellcheck = cs_CZ
\documentclass[a4paper]{article}
\usepackage[english]{babel}
\usepackage[utf8x]{inputenc}
\usepackage[T1]{fontenc}
\usepackage{listings}
\usepackage[a4paper,margin=2.1cm]{geometry}
\usepackage{amsmath}
\usepackage{graphicx}
\usepackage[colorlinks=true, allcolors=blue]{hyperref}
\usepackage[bottom]{footmisc}
%\setlength\parindent{0pt} % indent

\hyphenation{thatshouldnot}

\begin{document}

\title{NLP Applications 4\\Kateřina Veselovská, Forensic linguistics}
\author{Vilém Zouhar}
\date{Mar 2019}
\maketitle 

\section*{Introduction}

The topic of forensic linguistics and related issues was presented by Kateřina Veselovská from one of the biggest accounting companies, Deloitte. She is a alumni of our faculty with interest in sentiment analysis and works mostly with unstructured data. Her most common work tasks include text/risk and forensic analysis.

\section*{Forensic linguistics}

According to her, the field of forensic linguistics is rather new (emerged in 1960' and 1970' in Europe) and lacks formalized approaches and unified methodology. Kateřina Veselovská went briefly through tasks associated to this field. This includes text analysis, forensic psychology, forensic medicine, forensic phonetics and language material collection. I was surprised, that she did not mention Radek Skarnitzl, since he is well known for his work on forensic phonetics. 

\section*{Author identification}

She devoted a significant portion of time to the topic of author identification. In the domain of forensics this is used in cases of threat letters, blackmail letters, sexual harassement, informing calls, racketeering, law document intervention, etc. This was all well structured and very interesting. I was even taken aback by all of the attributes that can be mined (gender, age, social status, level of education, level of aggression, relationship to and relevance of other people, physical/mental handicap).

\section*{Other tasks}

The lecturer went on with several anecdotes from their company about solving disputes of meaning and use, plagiarism, voice identification, first language identification, social network (background checks, illegal content, fake news, hate speech). Even though each topic was interesting and she is a capable speaker, it soon became just an enumeration of topics.

\section*{Data mining}

The rest of the lecture was somewhat connected to data mining and mostly text processing. She spent several minutes on the later topics, which to me sounded only like glorified textual feature extraction. Kateřina Veselovská familiarized us with some basic issues related to text processing and e-Discovery. Mostly they have to go through huge volumes of unprocessed data (increase relevance) and be smart about the queries they use (because of the amount of fale positives they usually get). At the end she admitted, that some of their success in forensic was due to the element of surprise. People usually don't know how much can be infered from their data.

\section*{Summary}

The topic was probably the most interesting of all presented so far and the verbal presentation was very neat. Overall the lecture seemed well prepared, balanced and organized as opposed to some of the previous ones.

\end{document}