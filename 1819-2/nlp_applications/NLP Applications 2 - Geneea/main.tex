% !TeX spellcheck = cs_CZ
\documentclass[a4paper]{article}
\usepackage[english]{babel}
\usepackage[utf8x]{inputenc}
\usepackage[T1]{fontenc}
\usepackage{listings}
\usepackage[a4paper,margin=2cm]{geometry}
\usepackage{amsmath}
\usepackage{graphicx}
\usepackage[colorlinks=true, allcolors=blue]{hyperref}
\usepackage[bottom]{footmisc}
%\setlength\parindent{0pt} % indent

\begin{document}

\title{NLP Applications 2\\Jiří Hana, Geneea}
\author{Vilém Zouhar}
\date{Mar 2019}

\maketitle 

\vspace{-1cm}

\section*{Introduction}

The second NLP Applications lecture was about the Geneea company and was given by Jiří Hana. He started off with a comment on the perception of Czech on NLP Application (eg. misconception, that NLP stands only for Neuro Linguistics Programming). As oppossed to people in English speaking countries, the Czechs were in his experience mostly surprised by what Natural Language Processing has to offer.

\section*{Demos}
Jiří Hana mentioned presentation demos several times. He admitted to always being stressed when he is to give one (everything usually goes wrong, lack of concentration, etc..), but also accented the importance of well-thought presentations. According to his words, it is important to show something real (eg. a real user on the web in real browser, real URL), but at the same time have the demo simple and with a clear idea.

I found this part of the lecture quite helpful, because I never realized, that most of the time the demos are the only thing the client understands and can prove the marketer's claim for accessible, robust and easy to use software. He also sadly mentioned, that although they have access to huge amounts of data, they are not allowed to use them for presentation purposes, so they have to use data from the public domain. He accompanied this with several captivating graphs extracted from The Good Soldier Schweik.

\section*{Technical background}
The lecturer interested himself in technicalities only shallowly, which I think is a pity. He commented on their decision to use classical machine learning algorithms instead of deep learning, which they are forced to use in their marketing talks, since it has become a widespread buzzword. His arguments for their choice of approaches were common ones (lack of big annotated data, auditability, interpretability, computational power).

\section*{Media houses}
Even though this wasn't a technical part, it sounded interesting and it is a shame that he commented on this part only briefly. He talked about the requirements media houses usually have (article taging, content linking, etc), but also the confusion in the industry (underspecified term \textit{sentiment analysis}) and the accuracy assumptions in comparison to academia.

\section*{Business}
The whole lecture was interwined with notes about starting a business, which I found rather redundant. I found most of the points trivial and distracting, even though they of course had far-reaching implications. Some of the notes include not starting a company with someone we do not know very well, making long term plans and considering pros and cons of having a company self-funded.

\section*{Summary}
Generally I did not enjoy this lecture. Even though NLP was still the main theme, it appeared to me, that it was mostly a startup presentation with some additional tips, tricks and know-hows. His stress on marketing, non-technical points and market exploration is understandable, but it was not as interesting as the previous lecture.

\end{document}