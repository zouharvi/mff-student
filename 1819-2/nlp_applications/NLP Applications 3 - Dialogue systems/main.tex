% !TeX spellcheck = cs_CZ
\documentclass[a4paper]{article}
\usepackage[english]{babel}
\usepackage[utf8x]{inputenc}
\usepackage[T1]{fontenc}
\usepackage{listings}
\usepackage[a4paper,margin=2.1cm]{geometry}
\usepackage{amsmath}
\usepackage{graphicx}
\usepackage[colorlinks=true, allcolors=blue]{hyperref}
\usepackage[bottom]{footmisc}
%\setlength\parindent{0pt} % indent

\hyphenation{thatshouldnot}

\begin{document}

\title{NLP Applications 3\\Sylvie Cinková, Dialogue systems}
\author{Vilém Zouhar}
\date{Mar 2019}
\maketitle 

\section*{Introduction}

Sylvia Cinková started this lecture with several definitions related to dialogue systems, such as \textit{a dialogue, automatic speech recognition, text to speech, etc..} and I really enjoyed this formal and rigid start. She went on with other classifications (task oriented/non-task oriented, command/menu/natural langugage based, text/spoken/graphical ui/multi-modal dialogue systems, etc..). We were then showcased several demos, which were compelling, but rather outdated, too long and of poor quality.

\section*{Related subjects}

The lecturer mentioned, that designing dialogue systems requires cooperations with experts from other fields, such as phonology and psychology. She mentioned the interesting concept of \textit{uncanny valley} and problems connected to getting training data for dialogue systems. Until this point I didn't realise that the process of getting golden standard data for this domain is a field of study by itself.

\section*{Questions}

Even though this is a very common problem across all linguistics subfield, I'd be interested in knowing issues related to irony. All dialogue systems use some NLU modules, which must cope with this phenomena frequent in day to day conversations. I'd also be interested in more details of error handling (error detection, prediction and recovery), but didn't get the chance to ask.

\section*{Approaches}

I enjoyed the overview of different approaches to dialogue systems. These include \textit{rule-based}, \textit{frame-based}, \textit{dialogue context model} or more statistical approaches, such as \textit{(partially observerd) markov decision process}, \textit{utility maximalization}, or \textit{supervised/reinforced} learning.

\section*{Companion project}

Sylvia Cinková ended the lecture with a short story about the Czech companion project for senior citizens, which was almost cancelled halfway through, despite having the best results in the group. This part sounded interesting, but from lecture attendant's point of view, this was only an anecdote without much conceptual or technical details.

\section*{Summary}

Even though I am deeply insterested in dialogue systems and didn't enroll in the relevant course this semestr only because other obligations, I was lost for most of the second part of this lecture. This was generally the case that I was quite confused by this lecture, because it was composed by seemingly random remarks on disparate parts of dialogue systems instead of some more systematic approach.

\end{document}