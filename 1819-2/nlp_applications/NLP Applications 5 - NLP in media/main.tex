% !TeX spellcheck = cs_CZ
\documentclass[a4paper]{article}
\usepackage[english]{babel}
\usepackage[utf8x]{inputenc}
\usepackage[T1]{fontenc}
\usepackage{listings}
\usepackage[a4paper,margin=2.1cm]{geometry}
\usepackage{amsmath}
\usepackage{graphicx}
\usepackage[colorlinks=true, allcolors=blue]{hyperref}
\usepackage[bottom]{footmisc}
%\setlength\parindent{0pt} % indent

\hyphenation{thatshouldnot}

\begin{document}

\title{NLP Applications 5\\Jindřich Libovický, Deep learning/ML/NLP in media}
\author{Vilém Zouhar}
\date{Mar 2019}
\maketitle 

\section*{Introduction}

This lecture was much shorter than the previous ones. It was given by Jindřich Libovický, an ÚFAL employee. The topic was a bit misleading at the begining, since its true meaning became obvious only about midway through. At the start, he made a few remarks on the development of machine translation (mostly at Google) in the past few years. At this point, it seemed as a machine translation foused lecture, but by that time he started showing article headlines about artificial intelligence, which were full of buzzwords and incorrect interpretations of technical facts.\footnote{Google’s AI translation tool seems to have invented its own secret internal language,\\Facebook Forced To Shut Down Their Artificial Intelligence Program Because The Bots Created A Language Humans Couldn't Understand} It was amusing, but far from anything new. Journalists are known for intentionally writing clickbait articles.

\section*{Statistics}

The next part was focused on describing graphs of the development of the usage of AI, ML, Neural networks and others among different groups, such as programmers, researchers and IT specialists. The figures were neatly done, but after some time this becase monotonous. He always mentioned a comment, which provided some historical explanation.

\section*{Data extraction}

The rest of the lecture was devoted to the process by which Jindřich Libovický arrived at the data in his figures. This included some in my opinion too technical details\footnote{archive.org; wayback-machine-downloader; html2text + regexp; SpaCy tokenizer + named entity extraction} about data extraction, but at the same time he clearly outlined the whole process of his experiment.

\section*{Summary}

This lecture wasn't perhaps as informative as the previous ones, but it was definietly the most relaxing one and made me want to hear from Jindřich Libovický more about practical and fun aspects of NLP. I have only one small negative remark: he spoke rather quietly and even though I sat in the second row, it was sometimes hard for me to understand what he was saying and hence follow the lecture.

\end{document}