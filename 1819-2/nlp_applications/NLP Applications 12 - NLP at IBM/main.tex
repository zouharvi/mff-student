% !TeX spellcheck = cs_CZ
\documentclass[a4paper]{article}
\usepackage[english]{babel}
\usepackage[utf8x]{inputenc}
\usepackage[T1]{fontenc}
\usepackage{listings}
\usepackage[a4paper,margin=2.1cm]{geometry}
\usepackage{amsmath}
\usepackage{graphicx}
\usepackage[colorlinks=true, allcolors=blue]{hyperref}
\usepackage[bottom]{footmisc}
%\setlength\parindent{0pt} % indent

\hyphenation{thatshouldnot}

\begin{document}

\title{NLP Applications 12\\NLP at IBM\\Jan Cuřín, Martin Čmejrek}
\author{Vilém Zouhar}
\date{May 2019}
\maketitle 
\hspace{-4cm}


\section*{Introduction}
Both Jan Cuřín and Martin Čmejrek are MFF doctorate alumni. They now do research at IBM.

\section*{Dialogue Systems}

I was very confused about these 10 minutes, where Jan Cuřín talked about their IBM services. On the other hand he managed to convey some key concepts in dialogue systems (scripted paths and scenarios, challenges in obtaining good data, layers: Intents, Entities, Dialog Flow/Tree).

We were shown a demo of one of their early mock project \textit{Chez Pepe Restaurant}. This helped us understand what exactly the IBM Dialogue Platform is for and roughly how it works. Details about the API in JSON (which is de facto standard nowadays) seemed rather redundant.

\section*{First part summary}

I was lost for most of the time of this lecture. The lecturers spoke quickly and the overall presentation lacked a clear outline. Generally the lecture sounded more like a commercial pitch for IBM than an educational lesson.

\section*{Machine translation}
Martin Čmejrek started with a brief overview of a paper about statistical machine translation \footnote{\href{https://www.aclweb.org/anthology/H94-1028}{https://www.aclweb.org/anthology/H94-1028}}. He then went into a great depth about the EMNLP workshop\footnote{\href{https://sites.google.com/site/20yearsofbitext/}{https://sites.google.com/site/20yearsofbitext/}} and a very crude approach to statistical machine translation.

The lecturer then went to to talk about their TransTac\footnote{Translation for Tactical Use} project (speech recognition, synthetis and translation in restricted conditions). Gradually he got to explaining more about how statistical machine translation was done a what were the key problems.

\section*{Translation of Technical Literature}
Because of the strong competition, IBM focuses on very specific segments of machine translation, such as Technical Literature. They make heavy use of translation memories and in case of machine translation domain adaptation methods).

\section*{Second part summary}
Generally the second part of this lecture (by Martin Čmejrek) was much better. It was easier to follow, even though some parts in the narrative still seemed quite random and confusing.


\end{document}